\chapter{Symmetric Markov Operator}

In this chapter, let $E$ be a Polish space that is a separable complete metric space and let $E$ be equipped with the Borel $\sigma$-field $\mathcal{F}$. Then the measure decomposition theorem implies that for any probability measure $\mu$ on the product $\sigma$-field $\mathcal{F} \otimes \mathcal{F}$ on $E \times E$ with $\mu_1 = \pi_1^{\#}\mu$, the first projection, then
\begin{equation*}
	\mu (dx,dy) = k(x,dy)\mu_1(dx)
\end{equation*}
for some probability transition kernel $k \colon E \times \mathcal{F} \sto [0,1]$. Let $(\Omega,\Sigma,\Pb)$ be a probability space.
\begin{rmk}
	Moreover, because of the existence of kernels, by Ionescu–Tulcea theorem, for any probability measure $\mu$ on $E^n$, there are kernels $k_i$ from $E^{i-1}$ to $E$ such that
	\begin{equation*}
		\mu(dx_1,dx_2,\cdots,dx_n) = \mu_1(dx_1)k_2(x_1,dx_2)k_3(x_1,x_2,dx_3)\cdots k_n(x_1,\cdots,x_{n-1},dx_n).
	\end{equation*}		
\end{rmk}
For now on any measure $\mu$ is assumed to be $\sigma$-finite.

\section{Markov Operator}

\begin{defn}
    A Markov operator $P$ on $(E,\mathcal{F})$ is a linear operator $P \colon \mathcal{B}_b(E) \sto \mathcal{B}_b(E)$ such that
    \begin{enumerate}[label=(\arabic{*})]
    	\item (\emph{Mass conversation}) $P\mathds{1} = \mathds{1}$ for constant function $\mathds{1}(x) \equiv 1$,
		\item (Positivity preserving) for $f \geq 0$, $Pf \geq 0$.
    \end{enumerate}
\end{defn}
\begin{rmk}
    For $0 \leq f \leq 1$,
    \begin{equation*}
    	P(\mathds{1}-f) \geq 0~\Rightarrow~0 \leq P f \leq P \mathds{1} \leq \mathds{1}.
    \end{equation*}
    Therefore, $\|P f\|_{\infty} \leq\|f\|_{\infty}$ for all $f \in \mathcal{B}_b(E)$.
\end{rmk}

\begin{prop}[Jensen's inequality]
    For any convex $\phi \colon \R \sto \R$ and any $f \in \mathcal{B}_b(E)$, if $P$ is a Markov operator, then
    \begin{equation*}
        P(\phi(f)) \geq \phi(Pf)
    \end{equation*}
\end{prop}
\begin{proof}
    Because $\phi$ is convex, for any $b \in \R$, there is $a = a(b)$ such that
    \begin{equation*}
        \phi(c) \geq \phi(b) + a(b)(c - b),\quad \forall~c \in \R
    \end{equation*}
    For any $x \in E$, let $c = f(x)$. We have
    \begin{equation*}
        \phi(f(x)) \geq \phi(b) + a(b)(f(x) - b)\quad \Rightarrow \quad \phi(f) \geq \phi(b) + a(b)(f - b)
    \end{equation*}
    By the positivity and mass properties of $P$, we have
    \begin{equation*}
        P(\phi(f)) \geq \phi(b) + a(b)(Pf - b)
    \end{equation*}
    So for any $x \in E$,
    \begin{equation*}
         P(\phi(f))(x) \geq \phi(b) + a(b)(Pf(x) - b)
    \end{equation*}
    Then let $b = Pf(x)$, we get
    \begin{equation*}
        P(\phi(f))(x) \geq \phi(Pf(x))
    \end{equation*}
    which is true for any $x \in E$.
\end{proof}

\begin{defn}[Invariant Measure]
    A measure $\mu$ on $(E,\mathcal{F})$ is called invariant for a Markov operator $P$ if
    \begin{equation*}
    	\int_E Pf d\mu = \int_E f d\mu,
    \end{equation*}
    for all $f \in \mathcal{B}_b(E)$.
\end{defn}
\begin{rmk}
    When $f \in \mathcal{B}_b(E)$ is $0$ $\mu$-a.e., $Pf = 0$ $\mu$-a.e.. Therefore, $P$ can be extended on $L^\infty(\mu)$. Moreover, $\mu$ is invariant for $P$ if
    \begin{equation*}
    	\int_E Pf d\mu = \int_E f d\mu,\quad \forall~f \in L^1(\mu) \cap L^\infty(\mu).
    \end{equation*}
\end{rmk}

\noindent Note that for $1 \leq p < \infty$, $L^1(\mu) \cap L^\infty(\mu) \subset L^\infty(\mu)$ that is because
\begin{equation*}
	\|f\|_p^p=\int|f|^p d \mu \leq\|f\|_{\infty}^{p-1} \int|f| d \mu=\|f\|_{\infty}^{p-1}\|f\|_1.
\end{equation*}
So by Jensen's inequality for $\phi(x) = \abs{x}^p$ ($1 \leq p <\infty$),
\begin{equation*}
	\int|P f|^p d \mu \leq \int P\left(|f|^p\right) d \mu=\int|f|^p d \mu,\quad \forall~ f \in L^1(\mu) \cap L^\infty(\mu).
\end{equation*}
\begin{lem}
    For any $1 \leq p < \infty$,
    \begin{equation*}
    	L^1(\mu) \cap L^\infty(\mu) \subset L^p(\mu)
    \end{equation*}
    is dense.
\end{lem}
\begin{proof}
	Let $f \in L^p$ and $\varepsilon > 0$.
	\begin{itemize}
	    \item Step $1$: For any $n \in \N$, let 
	    \begin{equation*}
	    	g_n(x) \defeq \max\bc{-n,\min(f(x),n)} \in [-n,n].
	    \end{equation*}
	    Therefore, $g_n \in L^\infty$ and
	    \begin{equation*}
	    	\left\|f-g_n\right\|_p^p=\int_{|f|>n}| | f|-n|^p d \mu \leq \int_{|f|>n}|f|^p d \mu \underset{n \rightarrow \infty}{\longrightarrow} 0.
	    \end{equation*}
	    So let $n$ be sufficiently large such that $\left\|f-g_n\right\|_p \leq \varepsilon /2$.

	    \item By $\sigma$-finiteness, choose $E_k \uparrow E$ with $\mu(E_k) < \infty$ and put $h_k = g_n\mathbb{I}_{E_k}$. So $h_k \in L^1 \cap L^\infty$. Because $\mathbb{I}_{E_k^x} \sto 0$ as $k\sto \infty$, by DCT,
	    \begin{equation*}
	    	\left\|g_n-h_k\right\|_p^p=\int_{E_k^c}\left|g_n\right|^p d \mu \leq \int_{E_k^c}|f|^p d \mu \underset{k \rightarrow \infty}{\longrightarrow} 0.
	    \end{equation*}
	    Therefore,
	    \begin{equation*}
	    	\left\|f-h_k\right\|_p \leq\left\|f-g_n\right\|_p+\left\|g_n-h_k\right\|_p<\varepsilon. \qedhere
	    \end{equation*}
	\end{itemize}
\end{proof}
Then because
\begin{equation*}
	\norm{Pf}_p \leq \norm{f}_p,\quad \forall~ f \in L^1(\mu) \cap L^\infty(\mu),
\end{equation*}
and the density,
\begin{equation*}
	P \colon L^p(\mu) \sto L^p(\mu)
\end{equation*}
for all $1 \leq p \leq \infty$. Note that this definition should fix an invariant $\mu$.

\begin{defn}[Reversible Measure]
    A measure $\mu$ is called reversible for a Markov operator $P$ if 
    \begin{equation*}
        \int f P g d \mu=\int g P f d \mu,\quad \forall~f,g \in L^2(\mu).
    \end{equation*}
\end{defn}
\begin{rmk}
    It is obviously that if $\mu$ is reversible, then it is invariant, because it can choose $g_n \in L^2(\mu)$ such that $g_n \uparrow \mathds{1}$ by the $\sigma$-finiteness of $\mu$.
\end{rmk}

\begin{defn}
    A symmetric Markov semigroup on $(E, \mathcal{F},\mu)$ is a family of $(P_t)_{t \geq 0}$ of Markov operators such that
    \begin{enumerate}[label=(\roman*)]
    	\item (\emph{Initial Condition}) $P_0f = f$ for all $f \in L^\infty$;
    	\item (\emph{Semigroup}) for every $t,s \geq 0$, $P_tP_s = P_{t+s}$;
    	\item (\emph{Symmetry}) for every $t \geq 0$, $\mu$ is reversible for $P_t$;
    	\item (\emph{Strong Continuity}) for all $f \in L^2(\mu)$, $P_t f \sto f$ in $L^2(\mu)$ as $t \sto 0$.
    \end{enumerate}
\end{defn}
\begin{rmk}
    Note that strong continuity implies that $P_t \sto P_{t_0}$ in the strong operator topology on $L^2(\mu)$ as $t \sto t_0$ with the help of the initial condition and the semigroup property.
\end{rmk}

\begin{thm}[Kernel Representation]
    Let $P$ be a Markov operator on $(E,\mathcal{F})$ that is continuous on $L^1(\nu)$. Then there exists a probability kernel $p$ on $(E, \mathcal{F})$ such that for every $f \in L^\infty(\nu)$ and $\nu$-a.e. $x \in E$,
    \begin{equation*}
    	Pf(x) = \int_E f(y) p(x,dy).
    \end{equation*}
\end{thm}

\section{Generator}

For a given symmetric Markov semigroup $(P_t)_{t \geq 0}$ on $(E, \mathcal{F},\mu)$, we can similarly define the generator but the domain is different,
\begin{equation*}
	\mathcal{D}(L) \defeq \bb{f \in L^2(\mu)\colon \lim_{t \sto 0} \frac{Q_tf - f}{t} \text{ exists in } L^2(\mu)}.
\end{equation*}
And we also define $\mathcal{D}_p(L)$ for considering the convergence in $L^p(\mu)$. Except for the domain, some properties are as same as the generator of a Feller semigroup, like, $\mathcal{D}(L) \subset L^2(\mu)$ dense, and $L(P_t f) = P_t (L f)$. So
\begin{equation*}
	P_t f - f = \int_0^t P_s(Lf)ds = \int_0^t L(P_s f)ds.
\end{equation*}
Moreover, by the symmetry of $(P_t)_{t \geq 0}$, for any $f,g \in \mathcal{D}(L)$
\begin{equation*}
	\int_E f L g d \mu=\int_E g L f d \mu,
\end{equation*}
and for every $f \in \mathcal{D}_1(L)$,
\begin{equation*}
	\int L f d \mu=0.
\end{equation*}

\noindent Assume there exists an algebra $\mathcal{A} \subset \mathcal{D}(L)$, for example, $\mathcal{A} = C_c^\infty(\R^n)$.

\begin{defn}[Carr\'e du Champ]
    The carr\'e du champ associated to $L$ is the bilinear form $\Gamma$ on $\mathcal{A} \times \mathcal{A}$ defined by
    \begin{equation*}
    	\Gamma(f, g)=\frac{1}{2}(L(f g)-f L g-g L f).
    \end{equation*}
    $\Gamma(f) = \Gamma(f,f)$.
\end{defn}

Note that
\begin{equation*}
	\frac{d}{d t}\left(P_t f\right)^2=2 f \frac{d}{d t} P_t f=2 f L P_t f,
\end{equation*}
and by Jensen's inequality,
\begin{equation*}
	L\left(f^2\right)=\lim _{t \rightarrow 0} \frac{P_t\left(f^2\right)-f^2}{t} \geq \lim _{t \rightarrow 0} \frac{\left(P_t f\right)^2-f^2}{t}=\left.\frac{d\left(P_t f\right)^2}{d t}\right|_{t=0} \leq 2f Lf,
\end{equation*}
which implies that $\Gamma(f) \geq 0$. Then by the Cauchy-Schwartz inequality,
\begin{equation*}
	\Gamma(f,g)^2 \leq \sqrt{\Gamma(f)\Gamma(g)}.
\end{equation*}

\begin{prop}
    Let $(P_t)_{t \geq 0}$ on $(E, \mathcal{F},\mu)$ be a given symmetric Markov semigroup and $L$ be its generator. Then $L$ is a self-adjoint operator on $L^{\mu}$ and so it is closed.
\end{prop}

Moreover, because
\begin{equation*}
    0 \leq \int_E \Gamma(f) d\mu = -\int_E fLf d\mu,
\end{equation*}
$L$ is non-positive definite.



\paragraph{Construct semigroup from generator $L$.} Let's assume
\begin{equation*}
    L=\sum_{i, j=1}^n \sigma_{i j}(x) \frac{\partial^2}{\partial x_i \partial x_j}+\sum_{i=1}^n b_i(x) \frac{\partial}{\partial x_i},
\end{equation*}
where $b_i$ and $\sigma_{ij}$ are continuous functions and $\sigma = (\sigma_{ij}(x)) \in \R^{n \times n}$ is symmetric and nonnegative.  $\mathcal{D}(L) = C^\infty_c(\R^n)$. Moreover, of $\sigma$ is invertible, $L$ is called an elliptic diffusion operator.  A Borel measure $\mu$ is called symmetric for $L$ if for any $f,g \in C^\infty_c(\R^n)$,
\begin{equation*}
    \int_{\mathbb{R}^n} g L f d \mu=\int_{\mathbb{R}^n} f L g d \mu.
\end{equation*}
In the following, let's fix a measure $\mu$ symmetric for $L$.

Note that because $\mathcal{D}(L) = C^\infty_c(\R^n) \subset L^2(\R^n,\mu)$ is dense, $L$ is a non-positive symmetric operator that is densely defined on $L^2(\R^n,\mu)$. But it is not self-adjoint. However, it can be extended to a self-adjoint operator.
\begin{thm}[Friedrichs Extension]
    On the Hilbert space $L^2(\R^n,\mu)$, for $L$ defined above, there exists a densely defined non-positive self-adjoint extension of $L$.
\end{thm}

In fact, if $L$ is essentially self-adjoint, then the Friedrichs extension is the closed operator $\clo{L}$. In such case,
\begin{equation*}
    \ker (-L^* + \lambda I) = \bb{0},~\lambda > 0.
\end{equation*}
It means
\begin{equation*}
    -Lf + \lambda f = 0~\Rightarrow~f = 0,
\end{equation*}
where $Lf$, the differential in the sense of distribution.

Therefore, in the following, we assume $L$ is essentially self-adjoint and replace $\clo{L}$ by $L$. Then $L$ is self-adjoint on $L^2(\R^n,\mu)$. So we can define
\begin{equation*}
    P_t = e^{tL} = \int_{\R} e^{t\lambda} dE_L(\lambda) = \int_0^\infty e^{-t\lambda} dE_L(\lambda),\forall~t \geq 0,
\end{equation*}
where $E_L$ is the spectral measure associated with $L$. The $P_t \colon L^2(\R^n,\mu) \sto L^2(\R^n,\mu)$ is a bounded operator. Note that
\begin{enumerate}[label=(\roman*)]
	\item $P_tP_s = P_{t+s}$ for all $t,s \geq 0$.

    \item for all $f \in L^2$,
    \begin{equation*}
        \norm{P_t f}_{2} \leq \norm{f}_{2}.
    \end{equation*}

    \item for all $f \in L^2$, $t \mapsto P_tf$ is continuous in $L^2(\mu,\R^n)$.

    \item for all $f,g \in L^2$,
    \begin{equation*}
        \int_{\R^n} fP_tgd\mu = \int_{\R^n}gP_tf d\mu,
    \end{equation*}
    i.e., $\mu$ is reversible for $P_t$.

    \item for all $f \in L^2$,
    \begin{equation*}
    	 \lim_{t \sto 0} \norm{P_t f - f}_{2} = 0
    \end{equation*}

    \item for all any $f \in \mathcal{D}(L)$,
    \begin{equation*}
        \lim_{t \sto 0} \norm{\frac{P_t f - f}{t} - Lf}_{2} = 0.
    \end{equation*}

    \item if $\mathds{1} \in \mathcal{D}(L)$ and $L\mathds{1} = 0$, then $P_t\mathds{1} = \mathds{1}$.
\end{enumerate}

\section{Compact Markov Operators}

\begin{defn}[Diffusion Carr\'e du Champ]
    Let $\mathcal{A} \subset \R^E$ be an algebra such that for any $k \in \N$, any $f_1,\cdots,f_k \in \mathcal{A}$, and any $\Psi \in C^\infty(\R^k)$, $\Psi(f_1,\cdots,f_k) \in \mathcal{A}$. We say a bilinear form $\Gamma \colon \mathcal{A} \times \mathcal{A} \sto \mathcal{A}$ is called a diffusion carr\'e du champ if for any $\Psi$ and $f_i$ as above,
    \begin{equation*}
     	\Gamma\left(\Psi\left(f_1, \ldots, f_k\right), g\right)=\sum_{i=1}^k \partial_i \Psi\left(f_1, \ldots, f_k\right) \Gamma\left(f_i, g\right) .
    \end{equation*} 
\end{defn}

Consider a symmetric Markov semigroup with generator $L$ and the corresponding carr\'e du champ
\begin{equation*}
	\Gamma(f, g)=\frac{1}{2}(L(f g)-f L g-g L f).
\end{equation*}
If $\Gamma$ is a diffusion carr\'e du champ, then
\begin{equation}\label{eq:diffusion_generator}
	L \Psi\left(f_1, \ldots, f_k\right)=\sum_{i=1}^k \partial_i \Psi\left(f_1, \ldots, f_k\right) L f_i+\sum_{i, j=1}^k \partial_i \partial_j \Psi\left(f_1, \ldots, f_k\right) \Gamma\left(f_i, f_j\right).
\end{equation}
In particular, for $k = 1$,
\begin{align*}
	\Gamma(\psi(f), g) & =\psi^{\prime}(f) \Gamma(f, g) \\
	L \psi(f) & =\psi^{\prime}(f) L f+\psi^{\prime \prime}(f) \Gamma(f).
\end{align*}

\begin{defn}[Diffusion Semigroup]
    An operator $L$ satisfying (\ref{eq:diffusion_generator}) is called a diffusion generator. A symmetric Markov semigroup whose generator is a diffusion generator is called a diffusion semigroup.
\end{defn}

\begin{defn}[Dirichlet Form]
    A bilinear form $\mathcal{E} \colon \mathcal{D}(\mathcal{E}) \times \mathcal{D}(\mathcal{E}) \sto \R$ is called a Dirichlet form if
    \begin{enumerate}[label=(\roman*)]
    	\item $\mathcal{D}(\mathcal{E}) \subset L^2(\mu)$ dense for some $\mu$,
    	\item $\mathcal{E}(f,g) = \mathcal{E}(g,f)$ for $f,g \in \mathcal{D}(\mathcal{E})$,
    	\item $\mathcal{E}(f) = \mathcal{E}(f,f) \geq 0$ for $f \in \mathcal{D}(\mathcal{E})$,
    	\item $\mathcal{D}(\mathcal{E})$ is complete w.s.t. 
    	\begin{equation*}
    		\inn{f,g}_{\mathcal{E}} \defeq \int_Efgd\mu + \mathcal{E}(f,g),
    	\end{equation*}
    	\item for any $f \in \mathcal{D}(\mathcal{E})$, $0 \vee f \wedge 1 \in \mathcal{D}(\mathcal{E})$ and
    	\begin{equation*}
    		\mathcal{E}(0 \vee f \wedge 1) \leq \mathcal{E}(f).
    	\end{equation*}
    \end{enumerate}
\end{defn}
\begin{rmk}
    Note that for any symmetric, non-negative bilinear form $\mathcal{E}$ defined on some dense $D \subset L^2(\mu)$, if $\mathcal{E}$ satisfies that for any $f_n \sto 0$ in $L^2(\mu)$ and $f_n$ Cauchy w.s.t. $\inn{\cdot,\cdot}_{\mathcal{E}}$, $\mathcal{E}(f_n) \sto 0$, then $D$ can be extended to the closure of $D$ w.s.t. $\norm{\cdot}_2+\mathcal{E}(\cdot)$, and such $\mathcal{E}$ is called closable.
\end{rmk}

If $\Gamma$ is a diffusion carr\'e du champ on an algebra $\mathcal{A} \subset L^2(\mu)$ dense and $\Gamma(f,f) \geq 0$ for all $f \in \mathcal{A}$, then let
\begin{equation*}
	\mathcal{E}(f,g) \defeq \int_E \Gamma(f,g)d\mu
\end{equation*}
and taking $\mathcal{D}(\mathcal{E})$ be the closure of $\mathcal{A}$ w.s.t. $\inn{\cdot,\cdot}_{\mathcal{E}}$. It can prove that such $\mathcal{E}$ is a Dirichlet form. Moreover, by the symmetric and positivity of $\mathcal{E}$, Riesz representation theorem implies that we can define a a non-positive, symmetric operator $L$ by
\begin{equation*}
	\int g L f d \mu=-\mathcal{E}(f, g)
\end{equation*}
on the domain
\begin{equation*}
	\mathcal{D}(L)=\left\{f \in \mathcal{D}(\mathcal{E}): \exists C \text { such that } \mathcal{E}(f, g) \leq C\|g\|_2 \text { for all } g \in \mathcal{D}(\mathcal{E})\right\}.
\end{equation*}
Moreover, it can be extended to a self-adjoint operator $L$ by using Friedrichs extension.

\begin{defn}[Compact Markov Diffusion Triple]
    Let $(E,\mathcal{F},\mu)$ be a polished measure space and $\mu$ be a probability measure. For $\mathcal{A} \subset L^2(\mu)$, let
    \begin{equation*}
    	\Gamma \colon \mathcal{A} \times \mathcal{A} \sto \mathcal{A}
    \end{equation*}
    be a symmetric bilinear form. We say $(E,\mu,\Gamma)$ a compact Markov diffusion triple the followings are satisfied.
    \begin{enumerate}[label=(\alph*),series=myalph1]
    	\item $\mathcal{A}$ is dense in $L^2(\mu)$,
    	\item $\mathcal{A}$ is an algebra closed under composition with smooth functions,
    	\item $\Gamma(f) = \Gamma(f,f) \geq 0$ for all $f \in \mathcal{A}$,
    	\item $\Gamma$ is a diffusion carr\'e du champ,
    	\item $\Gamma(f) = 0$ implies that $f$ is a constant,
    \end{enumerate}
    and let $\mathcal{E}(f,g) = \int_E \Gamma(f,g)d\mu$ for all $f,g \in \mathcal{A}$, which satisfies
    \begin{enumerate}[label=(\alph*), resume=myalph1]
    	\item for every $f \in \mathcal{A}$, there exist a $C > 0$ such that $\mathcal{E}(f,g) \leq C\norm{g}_2$ for all $g \in \mathcal{A}$.
    \end{enumerate}
    It follows that $\mathcal{E}$ can be extended to a Dirichlet form. Let $L$ be the self-adjoint operator defined on $\mathcal{D}(L)$ such that
    \begin{equation*}
    	\int g L f d \mu=-\mathcal{E}(f, g).
    \end{equation*}
    Note that $\mathcal{A} \subset \mathcal{D}(L)$. Let $P_t = e^{tL}$ called the semigroup be assumed that
    \begin{enumerate}[label=(\alph*),resume=myalph1]
    	\item $L\mathcal{A} \subset \mathcal{A}$,
    	\item $P_t \mathcal{A} \subset \mathcal{A}$.
    \end{enumerate}
\end{defn}

\begin{prop}
    Let $(E,\mathcal{F},\mu)$ be a compact Markov diffusion triple and $P_t$ be its semigroup.
    \begin{enumerate}[label=(\arabic{*})]
    	\item $P_t$ is a symmetric Markov semigroup for $\mu$.
    	\item For any $f \in L^2(\mu)$,
    	\begin{equation*}
    		\lim_{t \sto \infty}P_tf = \int_E fd\mu
    	\end{equation*}
    	in $L^2$, which is called the ergodic property.
    \end{enumerate}
\end{prop}

\paragraph{Curvature.} 
\begin{defn}
   Given a compact Markov diffusion triple $(E,\mathcal{F},\mu)$. For any $f,g \in \mathcal{A}$,
   \begin{equation*}
    	\Gamma_2(f, g)=\frac{1}{2}\bc{L \Gamma(f, g)-\Gamma(f, L g)-\Gamma(g, L f)},
    \end{equation*}
    and $\Gamma_2(f) = \Gamma_2(f, f)$.
\end{defn}

\begin{defn}[Curvature Dimension]
    For $\rho \in \R$ and $n \in [1,\infty]$, a compact Markov diffusion triple $(E,\mathcal{F},\mu)$ is to said to satisfy the curvature-dimension condition $\op{CD}(\rho,n)$ if
    \begin{equation*}
    	\Gamma_2(f) \geq \rho \Gamma(f) + \frac{1}{n}(Lf)^2.
    \end{equation*}
\end{defn}

\section{Poincar\'e Inequality}

\begin{prop}
    Let $P_t$ be the semigroup of a compact Markov triple. TFAE.
    \begin{enumerate}[label=(\arabic{*})]
    	\item $\op{CD}(\rho,\infty)$ holds for some $\rho \in \R$.
    	\item For every $f \in \mathcal{A}$ and $t \geq 0$,
    	\begin{equation*}
    		\Gamma(P_tf) \leq e^{-2\rho t}P_t \Gamma(f).
    	\end{equation*}
    	\item For every $f \in \mathcal{A}$ and $t \geq 0$,
    	\begin{equation*}
    		P_t(f^2) - (P_tf)^2 \leq \frac{1 - e^{-2\rho t}}{\rho}P_t \Gamma(f).
    	\end{equation*}
    	\item For every $f \in \mathcal{A}$ and $t \geq 0$,
    	\begin{equation*}
    		P_t(f^2) - (P_tf)^2 \geq \frac{e^{2\rho t} - 1}{\rho}\Gamma(P_tf).
    	\end{equation*}
    \end{enumerate}
    For the last two conditions, if $\rho = 0$, then the coefficients in RHS can be taken as $2t$.
\end{prop}
\begin{proof}
    $(1)~\Rightarrow~(2)$: for $f \in \mathcal{A}$, let
    \begin{equation*}
    	\Lambda(s) = e^{-2\rho}P_s\Gamma(P_{t-s}f).
    \end{equation*}
    Then by chain rule,
    \begin{equation*}
    	\Lambda^{\prime}(s)=2 e^{-2 \rho s} P_s\left(\Gamma_2\left(P_{t-s} f\right)-\rho \Gamma\left(P_{t-s} f\right)\right) \geq 0,
    \end{equation*}
    because of $\op{CD}(\rho,\infty)$. Therefore, $\Lambda(t) \geq \Lambda(0)$.

    $(2)~\Rightarrow~(3)$: Let
    \begin{equation*}
    	\Lambda(s)=P_s\left(P_{t-s} f\right)^2.
    \end{equation*}
    So $\Lambda^{\prime}(s)=2 P_s \Gamma\left(P_{t-s} f\right)$ and
    \begin{align*}
    	\begin{aligned}
			\Lambda(t)-\Lambda(0) & =2 \int_0^t P_s \Gamma\left(P_{t-s} f\right) d s \\
			& \leq 2 \int_0^t e^{-2 \rho(t-s)} P_t \Gamma(f) d s \\
			& =\frac{1-e^{-2 \rho t}}{\rho} P_t \Gamma(f).
		\end{aligned}
    \end{align*}

    $(2)~\Rightarrow~(4)$: Similarly, as above, by using $P_s \Gamma\left(P_{t-s} f\right) \geq e^{2 \rho s} \Gamma\left(P_t f\right)$,
    \begin{equation*}
    	\Lambda(t)-\Lambda(0) \geq 2 \int_0^t e^{2 \rho s} \Gamma\left(P_t f\right) d s=\frac{e^{2 \rho t}-1}{\rho} \Gamma\left(P_t f\right).
    \end{equation*}

    $(3)~\Rightarrow~(1)$: Note that for any $h \in \mathcal{A}$,
    \begin{equation*}
    	P_t h=h+t L h+\frac{t^2}{2} L^2 h+o\left(t^2\right),\quad t \sto 0.
    \end{equation*}
    Therefore choosing $h = f$ and $h = f^2$, we have
    \begin{align*}
    	\begin{aligned}
			P_t\left(f^2\right)-\left(P_t f\right)^2 & =t L\left(f^2\right)+\frac{t^2}{2} L^2\left(f^2\right)-2 t f L f-t^2(L f)^2-t^2 f L^2 f+o\left(t^2\right) \\
			& =2 t \Gamma(f)+\frac{t^2}{2} L^2\left(f^2\right)-t^2(L f)^2-t^2 f L^2 f+o\left(t^2\right).
		\end{aligned}
    \end{align*}
    On the other hand,
    \begin{equation*}
    	\frac{1-e^{-2 \rho t}}{\rho} P_t \Gamma(f)=2 t \Gamma(f)-2 \rho t^2 \Gamma(f)+2 t^2 L \Gamma(f)+o\left(t^2\right).
    \end{equation*}
    Therefore, by $(3)$,
    \begin{equation*}
    	\frac{1}{2} L^2\left(f^2\right)-(L f)^2- f L^2 f+o(1) \leq -2 \rho \Gamma(f)+2 L \Gamma(f)+o(1).
    \end{equation*}
    As $t \sto 0$, we have
    \begin{equation*}
    	\frac{1}{2} L^2\left(f^2\right)-(L f)^2- f L^2 f \leq -2 \rho \Gamma(f)+2 L \Gamma(f).
    \end{equation*}
    Then by arranging,
    \begin{equation*}
    	L \Gamma(f)-2 \Gamma(f, L f) \geq 2 \rho \Gamma(f).
    \end{equation*}

    $(4)~\Rightarrow~(1)$: It is similarly as above. \qedhere

\end{proof}

For $(3)$ in above proposition, which is called local Poincar\'e inequality, if $\rho > 0$, by the ergodic property, as $t \sto \infty$,
\begin{equation*}
	\int f^2 d \mu-\left(\int f d \mu\right)^2 \leq \frac{1}{\rho} \int \Gamma(f) d \mu,
\end{equation*}
which is called a Poincar\'e inequality.

\begin{defn}[Poincar\'e inequality]
    Let $\mu$ be a probability measure and $\mathcal{E}$ be a Dirichlet form on $L^2(\mu)$. We say that $\mu$ and $\mathcal{E}$ satisfy a Poincar\'e inequality with constant $C$ ($\op{PI}(C)$) if
    \begin{equation*}
    	\int_E f^2 d\mu - \bc{\int_E fd\mu}^2 \leq C\mathcal{E}(f),
    \end{equation*}
    for any $f \in \mathcal{D}(\mathcal{E})$. The Poincar\'e constant of $\mu$ and $\mathcal{E}$ is the smallest $C$ such that above inequality holds for all $f \in \mathcal{D}(\mathcal{E})$.
\end{defn}
\begin{rmk}
    When considering a compact Markov triple, because $\mathcal{A} \subset \mathcal{D}(\mathcal{E})$ is dense, it suffices to check $\op{PI}$ on $\mathcal{A}$. Moreover, if a compact Markov triple satisfies $\op{CD}(\rho,\infty)$, it satisfies $\op{PI}(1/\rho)$.
\end{rmk}
\begin{cor}
    The compact Markov triple $(E,\mu,\Gamma)$ satisfies $\op{CD}(\rho,\infty)$ if and only if for any $t \geq 0$ and $x \in E$ $\mu$-a.e., the measure $p_t(x,\cdot)$ satisfies $\op{PI}$ with constant $(1 - e^{-2\rho t})/\rho$.
\end{cor}

\begin{prop}[Spectral Gap]
    If the compact Markov triple $(E,\mu,\Gamma)$ satisfies $\op{PI}(C)$ for some constant $C$, then the spectrum of $L$
    \begin{equation*}
    	\sigma(L) \subset (-\infty,-\frac{1}{C}] \cup \bb{0}.
    \end{equation*}
\end{prop}
\begin{proof}
    Let $\lambda \in \sigma(L)$ such that $\lambda \neq 0$. Because $L$ is self-adjoint, i.e., $\sigma(L) = \sigma_{ap}(L)$, there exists $f_n \in \mathcal{D}(L)$ such that $\norm{f}_2 = 1$ and
    \begin{equation*}
    	\norm{Lf_n - \lambda f_n}_2 \sto 0,\quad n \sto \infty.
    \end{equation*}
    Note that because $\mu$ is a probability measure, $\norm{Lf_n - \lambda f_n}_1 \sto 0$. It follows that $\int f_n d\mu \sto 0$ by $\int Lf_n d\mu = 0$ for all $n$. Then $\op{PI}$ implies that
    \begin{equation*}
    	\int_E f_n^2 d \mu-\left(\int_E f_n d \mu\right)^2 \leq C \int_E \Gamma\left(f_n\right) d \mu=-C \int_E f_n L f_n d \mu.
    \end{equation*}
    As $n\sto \infty$,
    \begin{equation*}
    	\lambda = \int_E f_n L f_n d \mu \leq -\frac{1}{C}. \qedhere
    \end{equation*}
\end{proof}

\paragraph{$\op{PI}$ under $\op{CD}(\rho,n)$.} 

\begin{lem}
    Suppose the compact Markov triple $(E,\mu,\Gamma)$ satisfies $\op{CD}(\rho,\infty)$ with some $\rho > 0$. It satisfies $\op{PI}(C)$ for some $C > 0$ if and only if
    \begin{equation*}
    	\int_E \Gamma(f) d \mu \leq C \int_E(L f)^2 d \mu,\quad \forall~ f\in \mathcal{D}(L).
    \end{equation*}
\end{lem}
\begin{proof}
    $\Rightarrow:$ Let
    \begin{equation*}
    	\Lambda(t)=\int_E\left(P_t f\right)^2 d \mu.
    \end{equation*}
    Then
    \begin{equation*}
    	\Lambda^{\prime}(t)=-2 \int_E \Gamma\left(P_t f\right) d \mu,\quad \Lambda^{\prime \prime}(t)=4 \int_E\left(L P_t f\right)^2 d \mu.
    \end{equation*}
    Because it satisfies $\op{CD}(\rho,\infty)$ with some $\rho > 0$, by
    \begin{equation*}
    	\Gamma(P_tf) \leq e^{-2\rho t}P_t \Gamma(f) \leq P_t \Gamma(f) \leq \Gamma(f)
    \end{equation*}
    Then by DCT, $\lim_{t \sto \infty} \Lambda^\prime(t)$ exists. And by ergodicity,
    \begin{equation*}
    	\lim_{t \sto \infty} \Lambda(t) = \int_E f d\mu,
    \end{equation*}
    $\lim_{t \sto \infty} \Lambda^\prime(t) = 0$. By assumption,
    \begin{equation*}
    	\Lambda^{\prime \prime}(t) \geq-\frac{2}{C} \Lambda^{\prime}(t).
    \end{equation*}
    Therefore,
    \begin{align*}
    	\begin{aligned}
			\int f^2 d \mu-\left(\int f d \mu\right)^2 & =-\int_0^{\infty} \Lambda^{\prime}(t) d t \\
			& \leq \frac{C}{2} \int_0^{\infty} \Lambda^{\prime \prime}(t) d t \\
			& =-\frac{C}{2} \Lambda^{\prime}(0) \\
			& =C \int_E \Gamma(f) d \mu.
		\end{aligned}
    \end{align*}

    $\Leftarrow:$ Choosing $f \in \mathcal{D}(L)$ with mean $0$ (otherwise, let $f - \int f d\mu$ and note that $\Gamma(c,g) = 0$ for any constant $c$ and function $g$). By Cauchy-Schwartz inequality,
    \begin{align*}
		\int_E \Gamma(f) d \mu & =\int_E f(-L f) d \mu \\
		& \leq \sqrt{\int_E f^2 d \mu \int_E(L f)^2 d \mu} \\
		& \leq \sqrt{C \int_E \Gamma(f) d \mu \int_E(L f)^2 d \mu}. \qedhere
    \end{align*}
\end{proof}

\begin{thm}
    Let $(E,\mu,\Gamma)$ be a compact Markov triple. If it satisfies $\op{CD}(\rho,n)$ for some $\rho > 0$ and $n > 1$, then $\mu$ satisfies $\op{PI}(C)$ with $C =\frac{n-1}{\rho n}$.
\end{thm}
\begin{proof}
    Because of $\op{CD}(\rho,n)$,
    \begin{equation*}
    	\int_E \Gamma_2(f) d \mu \geq \rho \int_E \Gamma(f) d \mu+\frac{1}{n} \int_E(L f)^2 d \mu.
    \end{equation*}
    Because $\int Lh d\mu = 0$,
    \begin{align*}
		\int_E \Gamma_2(f) & =\frac{1}{2} \bc{\int_E L \Gamma(f) d \mu-\int_E \Gamma(f, L f) d \mu} \\
		& =\frac{1}{2} \int_E L \Gamma(f) d \mu-\frac{1}{2} \int_E L(f L f) d \mu+\frac{1}{2} \int_E(L f)^2+f L^2 f d \mu \\
		& =\int_E(L f)^2 d \mu.
    \end{align*}
    Therefore,
    \begin{equation*}
    	\frac{n-1}{\rho n} \int_E(L f)^2 d \mu \geq  \int_E \Gamma(f) d \mu.
    \end{equation*}
    Then by above lemma, it has the result.
\end{proof}

\section{Applications with PI}

\paragraph{Decay of Variance.} For a probability measure $\mu$ and $f \in L^2(\mu)$, let
\begin{equation*}
	\operatorname{Var}_\mu(f)=\int_E f^2 d \mu-\left(\int_E f d \mu\right)^2.
\end{equation*}

\begin{prop}
    The compact Markov triple $(E,\mu,\Gamma)$ satisfies $\op{PI}(C)$ if and only if
    \begin{equation*}
    	\Var_\mu(P_t f) \leq e^{-\frac{2t}{C}}\Var_\mu(f),\quad f \in L^2(\mu).
    \end{equation*}
\end{prop}
\begin{proof}
    $\Rightarrow:$ For $f \in \mathcal{A}$,
    \begin{equation*}
    	\frac{d}{d t} \int_E\left(P_t f\right)^2 d \mu=2 \int_E P_t f L P_t f d \mu=-2 \mathcal{E}\left(P_t f\right).
    \end{equation*}
    Define
    \begin{equation*}
    	\Lambda(t)=e^{2 t / C} \operatorname{Var}_\mu\left(P_t f\right),
    \end{equation*}
    so
    \begin{equation*}
    	\Lambda^{\prime}(t)=\frac{2}{C} \operatorname{Var}_\mu\left(P_t f\right)-2 \mathcal{E}\left(P_t f\right) \leq 0
    \end{equation*}
    by $\op{PI}(C)$. It follows that $\Lambda(t) \leq \Lambda(0)$. For general $f \in L^2(\mu)$, it can get by density.

    $\Leftarrow:$ It suffices to prove that for $f \in \mathcal{A}$ with $\int f d\mu = 0$. Note that
    \begin{equation*}
    	P_t f=f+t L f+o(t),
    \end{equation*}
    and so
    \begin{equation*}
    	\operatorname{Var}\left(P_t f\right)=\int_E f^2 d \mu+2 t \int_E f L f d \mu+o(t).
    \end{equation*}
    On the other hand,
    \begin{equation*}
    	e^{-2 t / C} \operatorname{Var}_\mu(f)=\left(1-\frac{2 t}{C}+o(t)\right) \operatorname{Var}_\mu(f).
    \end{equation*}
    Therefore,
    \begin{equation*}
    	2 t \int_E f L f d \mu+o(t) \leq \left(-\frac{2 t}{C}+o(t)\right) \operatorname{Var}_\mu(f).
    \end{equation*}
    Then dividing $t$ on the both sides and taking $t \sto 0$,
    \begin{equation*}
    	2 \int_E f L f d \mu \leq-\frac{2}{C} \operatorname{Var}_\mu(f). \qedhere
    \end{equation*}
\end{proof}


\paragraph{Log-concave measures.}

\begin{defn}
    The probability measure $\mu$ on $\R^n$ defined by
    \begin{equation*}
    	d\mu(x) = e^{-W(x)}dx
    \end{equation*}
    is called log-concave if $W \colon \R^n \sto \R$ is convex. For $\rho > 0$, $\mu$ is called $\rho$-strongly log-concave if $W(x) - \rho \abs{x}^2$ is convex.
\end{defn}

Assume $W \in C^\infty(\R^n)$. And on $\R^n$,
\begin{equation*}
	\Gamma(f,g) = \inn{\nabla f, \nabla g}
\end{equation*}
is a carr\'e du champ. Then by divergence theorem,
\begin{equation*}
	-\int_{\mathbb{R}^n} \Gamma(f, g) d \mu=-\int_{\mathbb{R}^n}\left\langle e^{-W} \nabla f, \nabla g\right\rangle d x=\int_{\mathbb{R}^n} f(\Delta g-\langle\nabla W, \nabla g\rangle) d \mu.
\end{equation*}
Therefore,
\begin{equation*}
	L g=\Delta g-\langle\nabla W, \nabla g\rangle.
\end{equation*}
If all derivatives of $W(x)$ grow at most polynomially fast as $\abs{x} \sto \infty$, then $(\R^n,\mu,\Gamma)$ is a compact Markov triple with $\mathcal{A}$ being the class of smooth, bounded functions whose derivatives all vanish super-polynomially fast.

Moreover,
\begin{equation*}
	\Gamma_2(f, g)=\left\langle\nabla^2 f, \nabla^2 g\right\rangle+(\nabla f)^\top\left(\nabla^2 W\right)\nabla g,
\end{equation*}
By the strongly convexity of $W$,
\begin{equation*}
	\Gamma_2(f, f) \geq \rho \norm{f}_2 = \rho \Gamma(f),
\end{equation*}
i.e., $(\R^n,\mu,\Gamma)$ is $\op{CD}(\rho,\infty)$.

\begin{cor}
    Every $\rho$-strongly log-concave probability measure satisfies $\op{PI}(1 / \rho)$, i.e.,
    \begin{equation*}
    	\Var_\mu(f) \leq \frac{1}{\rho} \E_\mu[\norm{\nabla f}^2].
    \end{equation*}
\end{cor}


\section{Log-Sobolev Inequality}

\begin{thm}[Strong Gradient Bound]
    Let $(E,\mu,\Gamma)$ be a compact Markov triple that satisfies $\op{CD}(\rho,\infty)$. Then for every $f \in \mathcal{A}$ and $t \geq 0$,
    \begin{equation*}
        \sqrt{\Gamma(P_tf)} \leq e^{-\rho t}P_t\sqrt{\Gamma(f)}.
    \end{equation*}
\end{thm}
\begin{proof}
    First, assume $f > 0$. Fix $t$ and define
    \begin{equation*}
        \Lambda(s) = P_s\sqrt{\Gamma(P_{t-s}f)}.
    \end{equation*}
    by the chain rule for $L$,
    \begin{align*}
        \Lambda^{\prime}(s) & =P_s L \sqrt{\Gamma\left(P_{t-s} f\right)}+P_s \frac{\frac{d}{d s} \Gamma\left(P_{t-s} f\right)}{2 \sqrt{\Gamma\left(P_{t-s} f\right)}} \\
        & =P_s\left[\frac{L \Gamma\left(P_{t-s} f\right)}{2 \sqrt{\Gamma\left(P_{t-s} f\right)}}-\frac{\Gamma\left(\Gamma\left(P_{t-s} f\right)\right)}{4 \Gamma\left(P_{t-s} f\right)^{3 / 2}}-\frac{\Gamma\left(P_{t-s} f, L P_{t-s} f\right)}{\sqrt{\Gamma\left(P_{t-s} f\right)}}\right] \\
        & =P_s\left[\frac{\Gamma_2\left(P_{t-s} f\right)}{\sqrt{\Gamma\left(P_{t-s} f\right)}}-\frac{\Gamma\left(\Gamma\left(P_{t-s} f\right)\right)}{4 \Gamma\left(P_{t-s} f\right)^{3 / 2}}\right] .
    \end{align*}
    and so
    \begin{align*}
        \frac{d}{d s}\left(e^{-\rho s} \Lambda(s)\right) & =e^{-\rho s}\left(\Lambda^{\prime}(s)-\rho \Lambda(s)\right) \\
        & =e^{-\rho s} P_s\left[\frac{\Gamma_2\left(P_{t-s} f\right) \Gamma\left(P_{t-s} f\right)-\rho \Gamma\left(P_{t-s} f\right)^2-\frac{1}{4} \Gamma\left(\Gamma\left(P_{t-s} f\right)\right)}{4 \Gamma\left(P_{t-s} f\right)^{3 / 2}}\right].
    \end{align*}
    Let $g = P_{t-s}f$.
    \begin{equation*}
        \frac{d}{d s}\left(e^{-\rho s} \Lambda(s)\right) \geq 0 ~\Leftrightarrow~\Gamma(g)\left(\Gamma_2(g)-\rho \Gamma(g)\right) \geq \frac{1}{4} \Gamma(\Gamma(g)).
    \end{equation*}
    Therefore, it suffices to prove above inequality, which is followed by the diffusion property of $\Gamma$ and $\op{CD}(\rho,\infty)$.

    For general $f$, let $\psi(x) = \sqrt{x + \varepsilon}$ and replace $\Lambda(s)$ by
    \begin{equation*}
        \Lambda(s)=P_s \psi\left(e^{-2 \rho s} \Gamma\left(P_{t-s} f\right)\right). \qedhere
    \end{equation*}
\end{proof}

Let $\mu$ be a probability measure on $E$ and $f \colon E \sto [0,\infty)$ measurable. Define the entropy by
\begin{equation*}
    \Ent_\mu(f) \defeq \int_E f\log f d\mu - \int_E fd\mu \log \bc{\int_E f d\mu},
\end{equation*}
where we adopt the convention that $ 0 \log 0 =0$. By Jensen's inequality for $\psi(x) = x \log x$, $\Ent_\mu f \geq 0$. Because $\psi$ is strictly convex, $\Ent_\mu f = 0$ if and only if $f$ is constant $\mu$-a.e. Also,
\begin{equation*}
    \Ent_\mu(cf) = c\Ent_\mu(f),\quad \forall~c > 0.
\end{equation*}
\begin{rmk}
    Note that if $\nu \ll \mu$ is another probability measure and let $f = \frac{d \nu}{d \mu}$, 
    \begin{equation*}
        \Ent_\mu(f) = \int_E f\log f d\mu = \op{KL}(\nu ~\Vert~ \mu).
    \end{equation*}
\end{rmk}

\begin{defn}[Log-Sobolev Inequality]
    If $\mu$ is a probability measure and $\mathcal{E}$ is a Dirichlet form, we say they satisfy a log-Sobolev inequality with constant C ($\op{LSI}(C)$) if for all $f \in \mathcal{D}(\mathcal{E})$,
    \begin{equation*}
        \Ent_\mu(f^2) \leq 2C \mathcal{E}(f).
    \end{equation*}
    The smallest $C$ for which $\mu$ and $\mathcal{E}$ satisfy a $\op{LSI}(C)$ is called the log-Sobolev constant of $\mu,\mathcal{E}$.
\end{defn}

Assume $f > \varepsilon >0$, i.e., $f$ is bounded below. Then
\begin{equation*}
    \mathcal{E}(\sqrt{f}) = \int_E \Gamma(\sqrt{f})d\mu = \frac{1}{4}\int_E \frac{\Gamma(f)}{f} d\mu.
\end{equation*}
Then $\op{LSI}(C)$ is equivalent to
\begin{equation*}
    \operatorname{Ent}_\mu(f) \leq 2 C \mathcal{E}(\sqrt{f})=\frac{C}{2} \int_E \frac{\Gamma(f)}{f} d \mu.
\end{equation*}

\begin{defn}
    Let $\nu \ll \mu$ be another probability measure and $f = \frac{d \nu}{d \mu}$. The Fisher information of $\nu$ w.s.t. $\mu$ is defined as
    \begin{equation*}
        \mathrm{I}(\nu \mid \mu)=\mathrm{I}_\mu(f)=\int_E \frac{\Gamma(f)}{f} d \mu
    \end{equation*}
    and the entropy of $\nu$ w.s.t. $\mu$ (i.e., KL divergence) is defined as
    \begin{equation*}
        H(\nu \mid \mu) = \Ent_\mu(f).
    \end{equation*}
\end{defn}

By $\Ent_\mu(cf) = c\Ent_\mu(f)$, $\mu, \Gamma$ satisfy $\op{LSI}(C)$ if and only if
\begin{equation*}
    H(\nu \mid \mu) \leq \frac{C}{2}I(\nu \mid \mu)
\end{equation*}
for every probability measure $\nu \ll \mu$, where we allow infinity on the both sides. Moreover, if $\frac{d \nu}{d \mu} \notin \mathcal{D}(\mathcal{E})$, the RHS is $\infty$.

\begin{prop}
    If $\mu,\mathcal{E}$ satisfy $\op{LSI}(C)$, then they satisfy $\op{PI}(C)$.
\end{prop}
\begin{proof}
    Given $f \in \mathcal{A}$ with mean $0$ and $\varepsilon > 0$. Because $\log(1 + x) = x -\frac{1}{2}x^2 + o(x^2)$,
    \begin{align*}
        \Ent_\mu((1 + \epsilon f)^2) &= \int_E(1+\epsilon f)^2\left(\epsilon f-\frac{\epsilon^2}{2} f^2\right) d \mu-\int_E(1+\epsilon f)^2 d \mu \log \int_E(1+\epsilon f)^2 d \mu+o\left(\epsilon^2\right) \\
        &= 2 \epsilon^2 \int_E f^2 d \mu+o\left(\epsilon^2\right)
    \end{align*}
    Moreover,
    \begin{equation*}
        \mathcal{E}(1+\epsilon f) = \epsilon^2\mathcal{E}(f)
    \end{equation*}
    Apply $\op{LSI}$ to $1 + \epsilon f$.
    \begin{equation*}
        \mathcal{E}(1+\epsilon f) \leq 2C \epsilon^2\mathcal{E}(f).
    \end{equation*}
    Dividing $\epsilon$ and taking $\epsilon \sto 0$,
    \begin{equation*}
        \int_E f^2 d \mu \leq C \mathcal{E}(f). \qedhere
    \end{equation*}
\end{proof}

\paragraph{$\op{LSI}$ under $\op{CD}(\rho,\infty)$.}

\begin{prop}\label{prop:cd_lsi}
    For a compact Markov triple $(E,\mu,\Gamma)$, TFAE.
    \begin{enumerate}[label=(\arabic{*})]
        \item It satisfies $\op{CD}(\rho,\infty)$ for some $\rho \in \R$.
        \item For all $f \in \mathcal{A}$,
        \begin{equation*}
            \Gamma(f)\bc{\Gamma_2(f) - \rho\Gamma(f)} \geq \frac{1}{4}\Gamma(\Gamma(f)).
        \end{equation*}
        \item For every $f \in \mathcal{A}$ and $t \geq 0$,
        \begin{equation*}
            \sqrt{\Gamma(P_t f)} \leq e^{-\rho t}P_t \sqrt{\Gamma(f)}.
        \end{equation*}
        \item For every positive $f \in \mathcal{A}$ and $t \geq 0$,
        \begin{equation*}
            P_t(f \log f) - P_t f \log P_t f \leq \frac{1 - e^{-2\rho t}}{2\rho}P_t \frac{\Gamma(f)}{f}.
        \end{equation*}
        \item For every positive $f \in \mathcal{A}$ and $t \geq 0$,
        \begin{equation*}
            P_t(f \log f)-P_t f \log P_t f \geq \frac{e^{2 \rho t}-1}{2 \rho} \frac{\Gamma\left(P_t f\right)}{P_t f}.
        \end{equation*}
    \end{enumerate}
\end{prop}
Note that for $(4)$, it is called a local $\op{LSI}$. As $t \sto \infty$, by ergodicity, it obtains $\op{LSI}(1 / \rho)$.
\begin{cor}
   If $(E,\mu,\Gamma)$ is a compact Markov triple satisfies $\op{CD}(\rho,\infty)$ for some $\rho > 0$, then $\mu,\Gamma$ satisfy a $\op{LSI}(1 / \rho)$. 
\end{cor}
\begin{proof}[Proof of Proposition \ref{prop:cd_lsi}]
    $(1) ~\Rightarrow~ (2) ~\Rightarrow~ (3)$ is by the strong gradient bound. $(3)$ together with Jensen's inequality implies that
    \begin{equation*}
        \Gamma\left(P_t f\right) \leq e^{-2 \rho t} P_t \Gamma(f),
    \end{equation*}
    which follows that $(1)$.

    $(3)~\Rightarrow~(4)$: Fix $t > 0$ and define
    \begin{equation*}
        \Lambda(s)=P_s\left[P_{t-s} f \log P_{t-s} f\right] = P_s \psi\left(P_{t-s} f\right),
    \end{equation*}
    for $\psi(x) = x \log x$. So by $(3)$,
    \begin{align*}
        \Lambda^\prime(s) &=P_s\left[\psi^{\prime \prime}\left(P_{t-s} f\right) \Gamma\left(P_{t-s} f\right)\right] \\
        &= P_s \frac{\Gamma\left(P_{t-s} f\right)}{P_{t-s} f} \\
        &\leq e^{-2 \rho(t-s)} P_s \frac{\left(P_{t-s} \sqrt{\Gamma(f)}\right)^2}{P_{t-s} f},
    \end{align*}
    Note that by Cauchy-Schwartz inequality,
    \begin{equation*}
        \mathbb{E} X=\mathbb{E}\left[\sqrt{Y} \frac{X}{\sqrt{Y}}\right] \leq \sqrt{\mathbb{E} Y \mathbb{E} \frac{X^2}{Y}} ~\Rightarrow~ \frac{(\mathbb{E} X)^2}{\mathbb{E} Y} \leq \mathbb{E} \frac{X^2}{Y}.
    \end{equation*}
    By setting $X = \sqrt{\Gamma(f)},Y=f$ and taking expectation w.s.t. $p_{t-s}(x,\cdot)$,
    \begin{equation*}
        \Lambda^{\prime}(s) \leq e^{-2 \rho(t-s)} P_s P_{t-s} \frac{\Gamma(f)}{f}=e^{-2 \rho(t-s)} P_t \frac{\Gamma(f)}{f} .
    \end{equation*}
    So
    \begin{equation*}
        \Lambda(t)-\Lambda(0) \leq P_t \frac{\Gamma(f)}{f} \int_0^t e^{-2 \rho(t-s)} d s=\frac{1-e^{-2 \rho t}}{2 \rho} P_t \frac{\Gamma(f)}{f}.
    \end{equation*}

    $(3)~\Rightarrow~(5):$ It is similar as $(3)~\Rightarrow~(4)$, except for taking $X=\sqrt{\Gamma\left(P_{t-s} f\right)}, Y=P_{t-s} f$ and the expectation w.s.t. $p_s(x, \cdot)$, which impllies that
    \begin{equation*}
        \Lambda^{\prime}(s) \geq \frac{\left(P_s \sqrt{\Gamma\left(P_{t-s} f\right)}\right)^2}{P_t f} \geq e^{2 \rho s} \frac{\Gamma\left(P_t f\right)}{P_t f}.
    \end{equation*}

    $(4)~\Rightarrow~(1):$ $(4)$ is local $\op{LSI}(\rho)$, which implies local $\op{PI}$ as similar as above proposition. Then it implies $\op{CD}(\rho,\infty)$. $(5)~\Rightarrow~(1)$ is as similar as above.
\end{proof}

\paragraph{$\op{LSI}$ under $\op{CD}(\rho,n)$.}

\begin{thm}\label{thm:lsi_cdn}
    If a compact Markov triple $(E,\mu,\Gamma)$ satisfies $\op{CD}(\rho,n)$ for some $\rho > 0$, then $\mu,\Gamma$ satisfy a $\op{LSI}(C)$.
\end{thm}
\begin{lem}
    Suppose that
    \begin{equation*}
        \int_E f \Gamma(\log f) d \mu \leq C \int_E f \Gamma_2(\log f) d \mu
    \end{equation*}
    for some $C > 0$ and all positive $f \in \mathcal{A}$. Then $\mu,\Gamma$ satisfy a $\op{LSI}(C)$.
\end{lem}
\begin{proof}
    For fix some positive $f \in \mathcal{A}$. Let
    \begin{equation*}
        \Lambda(t)=\int_E P_t f \log P_t f d \mu.
    \end{equation*}
    Then
    \begin{equation*}
        \Lambda^{\prime}(t)=-\int_E \frac{\Gamma\left(P_t f\right)}{P_t f} d \mu=-\int_E P_t f \Gamma\left(\log P_t f\right) d \mu,
    \end{equation*}
    and
    \begin{equation*}
        \Lambda^{\prime \prime}(t)=\int_E \frac{\Gamma\left(P_t f\right) L P_t f}{\left(P_t f\right)^2}-\frac{2 \Gamma\left(P_t f, L P_t f\right)}{P_t f} d \mu.
    \end{equation*}
    Taking $g = P_t f$, by the diffusion property of $L$,
    \begin{equation*}
        L \frac{\Gamma(g)}{g}=-\frac{\Gamma(g) L g}{g^2}+\frac{L \Gamma(g)}{g}-\frac{2 \Gamma(g, \Gamma(g))}{g^2}+\frac{2 \Gamma(g)^2}{g^3} .
    \end{equation*}
    Since $\int Lh d\mu = 0$,
    \begin{equation*}
        \int_E \frac{\Gamma(g) L g}{g^2}=\int_E \frac{L \Gamma(g)}{g}-2 \frac{\Gamma(g, \Gamma(g))}{g^2}+2 \frac{\Gamma(g)^2}{g^3} d \mu .
    \end{equation*}
    Note that $L \Gamma(g)-2 \Gamma(g, L g)=2 \Gamma_2(g)$. So
    \begin{equation*}
        \Lambda^{\prime \prime}(t)=2 \int_E \frac{\Gamma_2(g)}{g}-\frac{\Gamma(g, \Gamma(g))}{g^2}+\frac{\Gamma(g)^2}{g^3} d \mu = \int_E g \Gamma_2(\log g).
    \end{equation*}
    Therefore,
    \begin{align*}
        \Lambda^{\prime}(t) & =-\int_E P_t f \Gamma\left(\log P_t f\right) d \mu \\
        \Lambda^{\prime \prime}(t) & =2 \int_E P_t f \Gamma_2\left(\log P_t f\right) d \mu.
    \end{align*}
    By assumption of $-\Lambda^{\prime}(t) \leq \frac{C}{2} \Lambda^{\prime \prime}(t)$, $\Lambda^\prime(t) \geq \Lambda^\prime(0) \exp(-\frac{2t}{C})$, i.e.,
    \begin{equation*}
        \int_E \frac{\Gamma\left(P_t f\right)}{P_t f} d \mu \leq e^{-2 t / C} \int_E \frac{\Gamma(f)}{f} d \mu.
    \end{equation*}
    Therefore,
    \begin{equation*}
        \Lambda(0)-\Lambda(t)=-\int_0^t \Lambda^{\prime}(s) d s \leq-\Lambda^{\prime}(0) \int_0^{\infty} e^{-2 s / C} d s=\frac{C\left(1-e^{-2 t / C}\right)}{2} \mathrm{I}_\mu(f),
    \end{equation*}
    then it can prove that by taking $t \sto \infty$.
\end{proof}
\begin{proof}[Proof of Theorem \ref{thm:lsi_cdn}]
    It suffices to check the condition of above lemma. By the diffusion property of $\Gamma$,
    \begin{align*}
        \Gamma_2\left(e^{a g}\right) & =a^2 e^{2 a g}\left[\Gamma_2(g)+a \Gamma(g, \Gamma(g))+a^2 \Gamma(g)^2\right] \\
        \Gamma\left(e^{a g}\right) & =a^2 e^{2 a g} \Gamma(g) \\
        L e^{a g} & =a e^{a g}[L g+a \Gamma(g)] .
    \end{align*}
    Therefore, $\op{CD}(\rho,n)$ implies that
    \begin{align*}
        \Gamma_2\left(e^{a g}\right) & =a^2 e^{2 a g}\left[\Gamma_2(g)+a \Gamma(g, \Gamma(g))+a^2 \Gamma(g)^2\right] \\
        \Gamma\left(e^{a g}\right) & =a^2 e^{2 a g} \Gamma(g) \\
        L e^{a g} & =a e^{a g}[L g+a \Gamma(g)] .
    \end{align*}
    So
    \begin{equation*}
        \int_E e^g\left[\Gamma_2(g)+a \Gamma(g, \Gamma(g))+a^2 \Gamma(g)^2-\rho \Gamma(g)-\frac{1}{n}[L g+a \Gamma(g)]^2\right] d \mu \geq 0 .
    \end{equation*}
    Note that
    \begin{equation*}
        4\left(L e^{g / 2}\right)^2=e^g\left[L g+\frac{1}{2} \Gamma(g)\right]^2,~\text{ and }~\int_E(L f)^2 d \mu=\int_E \Gamma_2(f) d \mu.
    \end{equation*}
    It implies that
    \begin{align*}
        & \int_E[L g+a \Gamma(g)]^2 d \mu \\
        & =\int_E 4\left(L e^{g / 2}\right)^2+e^g\left[(2 a-1) \Gamma(g) L g+\frac{4 a^2-1}{4} \Gamma(g)^2\right] d \mu \\
        & =\int_E 4 \Gamma_2\left(e^{g / 2}\right)+e^g\left[(2 a-1) \Gamma(g) L g+\frac{4 a^2-1}{4} \Gamma(g)^2\right] d \mu \\
        & =\int_E 4 \Gamma_2\left(e^{g / 2}\right)+e^g\left[(2 a-1) \Gamma(g) L g+\frac{4 a^2-1}{4} \Gamma(g)^2\right] d \mu \\
        & =\int_E e^g\left[\Gamma_2(g)+\frac{1}{2} \Gamma(g, \Gamma(g))+(2 a-1) \Gamma(g) L g+a^2 \Gamma(g)^2\right] d \mu,
    \end{align*}
    and so
    \begin{equation*}
        \int_E e^g \Gamma(g) L g d \mu=-\int_E \Gamma\left(g, e^g \Gamma(g)\right) d \mu=-\int_E e^g \Gamma(g, \Gamma(g))+e^g \Gamma(g)^2 d \mu.
    \end{equation*}
    Then the inequality implies that
    \begin{equation*}
        \int_E e^g\left[\frac{n-1}{n} \Gamma_2(g)+b_n \Gamma(g, \Gamma(g))+c_n \Gamma(g)^2-\rho \Gamma(g)\right] d \mu \geq 0
    \end{equation*}
    for
    \begin{equation*}
        b_n=\frac{2 a n+4 a-3}{2 n},\quad c_n=\frac{n a^2-(a-1)^2}{n} .
    \end{equation*}
    By choosing $a = \frac{3}{2n+4}$, it has
    \begin{equation*}
        \int_E e^g \Gamma_2(g) d \mu \geq \frac{n \rho}{n-1} \int_E e^g \Gamma(g) d \mu.\qedhere
    \end{equation*}
\end{proof}
   
\section{Applications with LSI}

\paragraph{Decay of Entropy.}

\begin{prop}
    The compact Markov triple $(E,\mu,\Gamma)$ satisfies a $\op{LSI}(C)$ if and only if
    \begin{equation*}
        \operatorname{Ent}_\mu\left(P_t f\right) \leq e^{-2 t / C} \operatorname{Ent}_\mu(f)
    \end{equation*}
    for every $t \geq 0$ and every $f \in L^{\mu}$ with finite entropy.
\end{prop}
\begin{proof}
    It suffices to consider $f \in \mathcal{A}$ with finite entropy. Define
    \begin{equation*}
        \Lambda(t)=\operatorname{Ent}_\mu\left(P_t f\right)=\int_E P_t f \log P_t f d \mu-\int_E  f d \mu \log \int_E  f d \mu.
    \end{equation*}

    $\Rightarrow:$ Note that
    \begin{equation*}
        \Lambda^{\prime}(t)=-\int_E \frac{\Gamma\left(P_t f\right)}{P_t f} d \mu=-I_\mu\left(P_t f\right),
    \end{equation*}
    so by $\op{LSI}$, $\Lambda^{\prime}(t) \leq-\frac{2}{C} \Lambda(t)$ that implies that $\Lambda(t) \leq e^{-2 t / C} \Lambda(0)$.

    $\Leftarrow:$ By Taylor expansion,
    \begin{equation*}
        \Lambda(t)=\Lambda(0)+t \Lambda^{\prime}(0)+o(t)=\Lambda(0)-t I_\mu(f)+o(t).
    \end{equation*}
    Because
    \begin{equation*}
        \Lambda(t) \leq e^{-2 t / C} \Lambda(0)=\left(1-\frac{2 t}{C}+o(t)\right) \Lambda(0),
    \end{equation*}
    as $t \sto 0$, we have
    \begin{equation*}
        I_\mu(f) \geq \frac{2}{C} \Lambda(0). \qedhere
    \end{equation*}
\end{proof}

If $f = \frac{d\nu_0}{d\mu}$, then for $\nu_t = P_t^*\mu_0$,
\begin{equation*}
    d\nu_t = P_t f d\mu.
\end{equation*}
Suppose $\mu,\Gamma$ satisfies $\op{LSI}(C)$. Then we have
\begin{equation*}
    H(\nu_t \mid \mu) \leq e^{-\frac{2t}{C}}H(\nu_0 \mid \mu).
\end{equation*}
Moreover, by the following Pinsker-Csizs\'ar-Kullback inequality,
\begin{equation*}
    d_{\op{TV}}(\mu,\nu_t)^2 \leq \frac{1}{2}e^{-\frac{2t}{C}}H(\nu_0 \mid \mu),
\end{equation*}
where
\begin{equation*}
    d_{\mathrm{TV}}(\mu, \nu)=\sup _{A \in \mathcal{F}}|\mu(A)-\nu(A)| = \frac{1}{2} \int_E\left|1-\frac{d \nu}{d \mu}\right| d \mu.
\end{equation*}
Moreover, $\op{PI}$ can also be applied to consider the convergence, becasue
\begin{equation*}
    d_{\mathrm{TV}}(\mu, \nu)^2 \leq \frac{1}{4} \operatorname{Var}_\mu\left(\frac{d \nu}{d \mu}\right),
\end{equation*}
but it needs $\frac{d \nu}{d \mu} \in L^2(\mu)$.

\begin{prop}[Pinsker-Csizs\'ar-Kullback]
    For any probability measure $\mu,\nu$ on the same space,
    \begin{equation*}
        d_{\mathrm{TV}}(\mu, \nu)^2 \leq \frac{1}{2} H(\nu \mid \mu),
    \end{equation*}
    where $H(\nu \mid \mu) = \infty$ if $\nu$ is not absolutely continuous to $\mu$.
\end{prop}
\begin{proof}
    WTLG, assume $f = \frac{d\nu}{d\mu} \in L^1$. Therefore, it suffices to show
    \begin{equation*}
        \left(\int_E|1-f| d \mu\right)^2 \leq 2 \operatorname{Ent}_\mu(f).
    \end{equation*}
    Define $f_s = 1 + s(f - 1)$ for $s \in [0,1]$ and
    \begin{equation*}
        \Lambda(s)=2 \operatorname{Ent}_\mu\left(f_s\right)-\left(\int_E\left|1-f_s\right| d \mu\right)^2=2 \operatorname{Ent}_\mu\left(f_s\right)-s^2\left(\int_E|1-f| d \mu\right)^2.
    \end{equation*}
    Since $\int f_s d\mu = 1$ for all $s$, it follows that
    \begin{equation*}
        \frac{d}{d s} \operatorname{Ent}_\mu\left(f_s\right)=\int_E(f-1)\left(1+\log f_s\right) d \mu,\quad \frac{d^2}{d s^2} \operatorname{Ent}_\mu\left(f_s\right)=\int_E \frac{(f-1)^2}{f_s} d \mu.
    \end{equation*}
    In particular, $\Lambda(0) = \Lambda^\prime(0) =0$ and
    \begin{equation*}
        \Lambda^{\prime \prime}(s)=2 \int_E \frac{(f-1)^2}{f_s} d \mu-2\left(\int_E|1-f| d \mu\right)^2 \geq 0,
    \end{equation*}
    by Cauchy-Schwartz inequality. Hence, $\Lambda(s) \geq 0$ for all $s \in [0,1]$.
\end{proof}

\paragraph{Hypercontractivity.} We already shown that if $\mu$ is an invariant measure, then $P_t \colon L^p(\mu) \sto L^p(\mu)$ is contractive.

\begin{thm}
    For a compact Markov triple with semigroup $P_t$, TFAE.
    \begin{enumerate}[label=(\arabic{*})]
        \item It satisfies $\op{LSI}(C)$.
        \item (Hypercontractivity) For some (or every) $1 < p < \infty$, every $t \geq 0$ and $f \in L^p(\mu)$,
        \begin{equation*}
            \norm{P_t f}_{q(t)} \leq \norm{f}_p,
        \end{equation*}
        where  $q(t)$ satisfies $\frac{q(t)-1}{p-1} = e^{2t / C}$.
        \item (Reverse hypercontractivity) For some (or every) $- \infty < p < 1$, every $t \geq 0$ and every positive, bounded $f \in \mathcal{A}$,
        \begin{equation*}
            \norm{P_t f}_{q(t)} \geq \norm{f}_p,
        \end{equation*}
        where $q(t)$ is as above.
    \end{enumerate}
\end{thm}
\begin{rmk}
    All eigenvectors of $L \colon L^2 \sto L^2$ belong to $L^q$ for all $2 < q <\infty$. If $Lf = -\lambda f$ for $\lambda > 0$ then $P_tf = e^{-\lambda t} f$. By above theorem with $p = 2$ and $t = \frac{C}{2}\log (q - 1)$,
    \begin{equation*}
        e^{-\lambda t}\|f\|_q=\left\|P_t f\right\|_q \leq\|f\|_2 ~\Rightarrow~ \|f\|_q \leq(q-1)^{C \lambda / 2}\|f\|_2.
    \end{equation*}
\end{rmk}

\section{Riemannian Markov Operator}

Let $(M,g)$ be a compact Riemannian manifold and $W \in C^\infty(M)$. WTLG, assume $\int_M e^{-W} dV = 1$, where $dV$ is the canonical volume form on $M$. Let $M$ be equipped with Borel $\sigma$-algebra and for any Borel set $A$, define
\begin{equation*}
    \mu(A) \defeq \int_M \mathbb{I}_Ae^{-W}dV,
\end{equation*}
which is a probability measure. Let $\mathcal{A} = C^\infty(M)$ and define $\Gamma \colon \mathcal{A} \times \mathcal{A} \sto \mathcal{A}$ by
\begin{equation*}
    \Gamma(f,g) = \inn{\nabla f,\nabla g}.
\end{equation*}
Locally, $\Gamma(f,g) = g^{ij}\partial_if \partial_j g$, which obviously a diffusion carr\'e du champ. We can obtain
\begin{equation*}
    Lg = \Delta g - \inn{\nabla W,\nabla g},\quad \forall~ g \in C^\infty(M),
\end{equation*}
because by the convergence theorem
\begin{equation*}
    \int_M f L g d \mu=-\int_M \Gamma(f, g) d \mu.
\end{equation*}
Moreover, 
\begin{equation*}
   \abs{\mathcal{E}(f,g)} = \abs{\int_M \Gamma(f,g) d\mu} \leq \norm{Lf}_{L^2(\mu)}\norm{g}_{L^2(\mu)},
\end{equation*}
which implies that $\mathcal{E}$ is closable. So it is a Dirichlet form.

\begin{thm}[Parabolic regularity theorem]
    Suppose that $u \colon M \times \R \sto \R$ is a bounded solution to
    \begin{equation*}
        \left\{
            \begin{aligned}
                \frac{\partial u(x,t)}{\partial t} &= \Delta u - \inn{\nabla W,\nabla u} \\
                u(x,0) &= f(x),
            \end{aligned}
        \right.
    \end{equation*}
    where $f,W \in C^\infty(M)$. Then $u(t,\cdot) \in C^\infty(M)$.
\end{thm}
Because $u(x,t) = (P_tf)(x)$ satisfies above PDE, $P_tf \in C^\infty(M)$ when $f \in C^\infty(M)$. Therefore, we have $P_t\mathcal{A} \subset \mathcal{A}$. It follows that $(M,\mu,\Gamma)$ is a compact Markov triple.

\begin{lem}
    Let $(M,g)$ be a Riemannian manifold with the Levi-Civita connection $\nabla$ and $f \in C^\infty(M)$.
    \begin{enumerate}[label=(\arabic{*})]
        \item Let $E_i$ be an orthonormal frame in $TM$. Then 
        \begin{equation*}
            \norm{\nabla^2 f}^2 = \sum_i \norm{\nabla_{E_i} (\nabla f)}^2,
        \end{equation*}
        where note that $\norm{\nabla^2 f}^2 = g^{kh}g^{lj}f_{;hl}f_{;kj}$.

        \item For any $p \in M$, choosing a normal coordinate centered $p$,
        \begin{equation*}
            (\Delta f)(p) = \sum_{i=1}^n (\partial_i^2 f)(p).
        \end{equation*}
    \end{enumerate}
\end{lem}
\begin{proof}
    \begin{enumerate}[label=(\arabic{*})]
        \item Locally, let
        \begin{equation*}
            E_i = E_i^\ell \frac{\partial}{\partial x^\ell}.
        \end{equation*}
        Let
        \begin{equation*}
            \nabla_{\frac{\partial}{\partial x^\ell}}dx^k = -\Gamma^k_{\ell h}dx^h,
        \end{equation*}
        where $\Gamma^k_{\ell h}$ is the Christoffel symbol. Then
        \begin{align*}
            \nabla_{E_i} (\nabla f) &= \nabla_{E_i} \bc{\frac{\partial f}{\partial x^k} dx^k} \\
            &= E_i^\ell \frac{\partial^2 f}{\partial x^k \partial x^\ell}dx^k - E^\ell_i\frac{\partial f}{\partial x^k}\Gamma^k_{\ell h}dx^h \\
            &= E_i^\ell \bc{\frac{\partial^2 f}{\partial x^h \partial x^\ell} - \frac{\partial f}{\partial x^k}\Gamma^k_{\ell h}}dx^h \\
            &= E_i^\ell f_{;h\ell} dx^h.
        \end{align*}
        So we have
        \begin{equation*}
            \norm{\nabla_{E_i} (\nabla f)}^2 = g^{kh}\bc{E_i^\ell f_{;h \ell}}\bc{E_i^j f_{;k j}},
        \end{equation*}
        and
        \begin{equation*}
            \sum_i\norm{\nabla_{E_i} (\nabla f)}^2 = g^{kh} \bc{ \sum_i E_i^\ell E_i^j} f_{;h \ell}f_{;k j}.
        \end{equation*}
        Because $E_i$ is orthonormal,
        \begin{align*}
            \inn{E_i,E_m} &= \inn{E_i^\ell \frac{\partial}{\partial x^\ell},E_m^j \frac{\partial}{\partial x^j}} \\
            &= E_i^\ell g_{\ell j} E_m^j = \delta_{im},
        \end{align*}
        which means that $E^\top gE = I$ for matrix $E = (E^m_j)_{m \times j}$. It follows that
        \begin{equation*}
            g^{-1} = EE^\top ~\Rightarrow~ g^{\ell j} = \sum_i E^\ell_iE^j_i.
        \end{equation*}
        Therefore,
        \begin{equation*}
             \sum_i\norm{\nabla_{E_i} (\nabla f)}^2 = g^{kh} g^{\ell j} f_{;h \ell}f_{;k j} = \norm{\nabla^2 f}^2.
        \end{equation*}

        \item Choose a normal coordinate centered at $p$, i.e. $g^{ij}(p) = \delta_{ij}$ and $\Gamma^h_{ij}(p) = 0$. Then
        \begin{align*}
            (\Delta f)(p) &= g^{ij}(p)f_{;ij}(p) \\
            &= g^{i j}(p)\left(\frac{\partial^2 f}{\partial x^i \partial x^j}(p)-\frac{\partial f}{\partial x^h}(p) \Gamma_{i j}^h(p)\right) \\
            &= \frac{\partial^2 f}{(\partial x^i)^2}(p). \qedhere
        \end{align*}
    \end{enumerate}
\end{proof}
\begin{rmk}
    In general, for $\omega_1,\omega_2 \in \Gamma(T^*M)$,
    \begin{equation*}
        \inn{\omega_1,\omega_2} \defeq \inn{\sharp \omega_1,\sharp \omega_2} = g(\sharp \omega_1,\sharp \omega_2).
    \end{equation*}
    Note that for $\omega_1 = f_i^1 dx^i$ and $\omega_2 = f_j^2 dx^j$,
    \begin{align*}
        \inn{\omega_1,\omega_2} &= \inn{g^{\ell i}f_i^1\frac{\partial}{\partial x^\ell},g^{k j}f_j^2\frac{\partial}{\partial x^k}} \\
        &= f_i^1f_j^2g^{\ell i}g^{k j} g_{ \ell k} \\
        &= f_i^1f_j^2g^{ji},
    \end{align*}
    which means that $\inn{\cdot,\cdot}$ on $T^*M$ with the matrix expression $g^{-1}$. In this notation,
    \begin{equation*}
        \inn{\omega,X} \defeq \inn{\sharp \omega, X} = \inn{\omega,\flat X} = \omega(X).
    \end{equation*}
    Note that $\inn{X,\op{grad}f} = X(f)$ and by above notation
    \begin{equation*}
        \inn{\omega ,\nabla f} = \inn{\omega, \op{grad} f} = \omega(\op{grad} f).
    \end{equation*}
    Moreover, for any $\omega \in T^*M$ and $X \in \Gamma(TM)$, because $\nabla$ is Levi-Civita,
    \begin{equation*}
        \nabla_X(\sharp \omega) = \sharp (\nabla_X \omega).
    \end{equation*}
    It is because for any $Y \in \Gamma(TM)$,
    \begin{align*}
        Xg(\sharp \omega,Y) &= g(\nabla_X \sharp \omega, Y) + g(\sharp \omega, \nabla_X Y) \\
        &= g(\nabla_X \sharp \omega, Y) + \omega(\nabla_X Y).
    \end{align*}
    On the other hand,
    \begin{align*}
        Xg(\sharp \omega,Y) = X(\omega(Y)) &= (\nabla_X \omega)(Y) + \omega (\nabla_X Y) \\
        &= g(\sharp (\nabla_X \omega),Y) + \omega (\nabla_X Y).
    \end{align*}
    Therefore,
    \begin{equation*}
        g(\sharp (\nabla_X \omega),Y) = g(\nabla_X \sharp \omega, Y)~\Rightarrow~ \nabla_X(\sharp \omega) = \sharp (\nabla_X \omega).
    \end{equation*}
    It follows that
    \begin{equation*}
        \inn{\nabla_X (\sharp \omega_1), \sharp \omega_2} = \inn{\nabla_X \omega_1,\omega_2},
    \end{equation*}
    which implies that
    \begin{equation*}
        \nabla_X \inn{\omega_1,\omega_2} = \inn{\nabla_X \omega_1,\omega_2} + \inn{\omega_1, \nabla_X \omega_2}.
    \end{equation*}
\end{rmk}

\begin{lem}
    Choose a orthonormal basis $\bb{E_i}$ and let
    \begin{equation*}
        H_{ij} = \nabla^2f(E_i,E_j).
    \end{equation*}
    Then
    \begin{equation*}
        \norm{\nabla^2 f}^2 = \sum_{i,j}H_{ij}^2,\quad \Delta f = \sum_j H_{ii},
    \end{equation*}
    i.e, $\norm{\nabla^2 f}^2 = \norm{H}_{\op{F}}^2$ and $\Delta f = \tr H$, which induces
    \begin{equation*}
        (\Delta f)^2 \leq n \norm{\nabla^2 f}^2
    \end{equation*}
    if $\dim M = n$.
\end{lem}
\begin{proof}
    First, for any $\omega \in \Gamma(T^*M)$,
    \begin{align*}
        \sum_j \omega(E_j)^2 &= \sum_j \bc{\sum_i \omega_i E^i_j}\bc{\sum_\ell \omega_\ell E^\ell_j} \\
        &= \sum_{i,\ell}  \omega_i \omega_\ell \bc{\sum_jE^i_jE^\ell_j} \\
        &= \sum_{i,\ell}  \omega_i \omega_\ell g^{i\ell} = \norm{\omega}^2.
    \end{align*}
    By above lemma,
    \begin{equation*}
        \norm{\nabla^2 f} = \sum_{i,j} \bc{\nabla^2 f (E_i,E_j)}^2.
    \end{equation*}
    And $\Delta f = \tr H$ is just by definition.
\end{proof}

\begin{defn}[Weighted Ricci Curvature]
    Given $W \in C^\infty(M)$, the weighted Ricci curvature is a $(2,0)$-tensor $\op{Ric}_W$ defined by
    \begin{equation*}
        \op{Ric}_W(X,Y) \defeq \op{Ric}(X,Y) + (\nabla^2 W)(X,Y).
    \end{equation*}
\end{defn}

\begin{prop}
    Let $(M,g)$ be a compact Riemannian manifold and $W \in C^\infty(M)$. Define
    \begin{align*}
        Lf &= \Delta f - \inn{\nabla W, \nabla f} \\
        \Gamma(f,g) &= \inn{\nabla f, \nabla g} \\
        \Gamma_2(f) &= \frac{1}{2}L\Gamma(f) - \Gamma(f,Lf).
    \end{align*}
    Then
    \begin{equation*}
        \Gamma_2(f) = \op{Ric}_W(\op{grad} f,\op{grad} f) + \norm{\nabla^2 f}^2.
    \end{equation*}
\end{prop}
\begin{proof}
    By Bochner’s formula, we have
    \begin{equation*}
        \frac{1}{2} \Delta \norm{\nabla f}^2 = \op{Ric}(\op{grad}f,\op{grad}f) + \inn{\nabla f, \nabla (\Delta f)} + \norm{\nabla^2 f}^2.
    \end{equation*}
    By definition,
    \begin{align*}
        \Gamma_2(f) &= \frac{1}{2} L (\norm{\nabla f}^2) - \inn{\nabla f, \nabla Lf} \\
        &= \frac{1}{2} \Delta (\norm{\nabla f}^2) - \frac{1}{2}\inn{\nabla W, \nabla (\norm{\nabla f}^2)} - \inn{\nabla f, \nabla (\Delta f)} + \inn{\nabla f, \nabla \inn{\nabla W, \nabla f}} \\
        &= \op{Ric}(\op{grad}f,\op{grad}f) + \norm{\nabla^2 f}^2  -  \frac{1}{2}\inn{\nabla W, \nabla (\norm{\nabla f}^2)} + \inn{\nabla f, \nabla \inn{\nabla W, \nabla f}}.
    \end{align*}
    Moreover,
    \begin{align*}
        \frac{1}{2}\inn{\nabla W, \nabla (\norm{\nabla f}^2)} &= \frac{1}{2}\nabla_{\op{grad} W} \inn{\nabla f, \nabla f} \\
        &= \inn{\nabla_{\op{grad} W}\nabla f, \nabla f}.
    \end{align*}
    On the other hand,
    \begin{align*}
        \inn{\nabla f, \nabla \inn{\nabla W, \nabla f}} &= \nabla_{\op{grad} f} \inn{\nabla W, \nabla f} \\
        &= \inn{\nabla_{\op{grad} f}\nabla W, \nabla f} + \inn{\nabla W, \nabla_{\op{grad} f}\nabla f} \\
        &= (\nabla^2 W)(\op{grad} f,\op{grad} f) + \inn{\nabla W, \nabla_{\op{grad} f}\nabla f},
    \end{align*}
    because
    \begin{equation*}
        (\nabla^2 W)(\op{grad} f,\op{grad} f) = (\nabla_{\op{grad} f} \nabla W)(\op{grad} f) = \inn{\nabla_{\op{grad} f} \nabla W, \nabla f}
    \end{equation*}
    Moreover,
    \begin{equation*}
        \inn{\nabla_{\op{grad} W}\nabla f, \nabla f} = \nabla^2f(\op{grad}f,\op{grad}W)
    \end{equation*}
    and
    \begin{equation*}
        \inn{\nabla W, \nabla_{\op{grad} f}\nabla f} = \nabla^2f(\op{grad}W,\op{grad} f),
    \end{equation*}
    they are same by the symmetry of $\nabla^2 f$. Therefore,
    \begin{align*}
        \Gamma_2(f) &= \op{Ric}(\op{grad}f,\op{grad}f) + \norm{\nabla^2 f}^2 + (\nabla^2 W)(\op{grad} f,\op{grad} f) \\
        &=\op{Ric}_W(\op{grad}f,\op{grad}f) + \norm{\nabla^2 f}^2. \qedhere
    \end{align*}
\end{proof}

\begin{cor}
    Consider a compact Markov triple $(M,\mu,\Gamma)$ defined as above, where $M$ has dimension $n$.
    \begin{enumerate}[label=(\arabic{*})]
        \item If $W \equiv c$ and $\op{Ric} \geq \rho g$, then $(M,\mu,\Gamma)$ satisfies $\op{CD}(\rho,n)$.
        \item If $\op{Ric}_W \geq \rho g$, then $(M,\mu,\Gamma)$ satisfies $\op{CD}(\rho,\infty)$.
        \item If $\op{Ric}_W \geq \rho g$ and $m > n$ such that
        \begin{equation*}
            \op{Ric}_W \geq \rho g + \frac{1}{m - n} \nabla W \otimes \nabla W,
        \end{equation*}
        then $(M,\mu,\Gamma)$ satisfies $\op{CD}(\rho,m)$.
    \end{enumerate}
\end{cor}
\begin{proof}
    \begin{enumerate}[label=(\arabic{*})]
        \item It is directly obtained by
        \begin{equation*}
            (\Delta f)^2 \leq n \norm{\nabla f}^2.
        \end{equation*}

        \item It is obvious.

        \item Similarly,
        \begin{align*}
            \Gamma_2(f) &\geq \rho \Gamma(f)+\frac{1}{m-n}\langle\nabla W, \nabla f\rangle^2+\frac{1}{n}(\Delta f)^2 \\
            &\geq \rho \Gamma(f)+\frac{1}{m}\bc{\Delta f - \langle\nabla W, \nabla f\rangle},
        \end{align*}
        where the last inequality is because $\frac{a^2}{m}+\frac{b^2}{m-n} \geq \frac{1}{m}(a+b)^2$. \qedhere
    \end{enumerate}
\end{proof}