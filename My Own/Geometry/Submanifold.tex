\chapter{Submanifolds}

\section{More for Distance Function}

\begin{enumerate}[label=\arabic{*}.]
	\item \emph{\textbf{Local Distance Function:}}	Suppose $(M,g)$ be a Riemannian manifold and $S \subset M$ be any subset, for each $\gamma(t) \in M$,
	\begin{equation*}
		d(x,S) \defeq \inf\bb{d(x,p) \colon p \in S}
	\end{equation*}
	Then we can see
	\begin{enumerate}[label=(\arabic{*})]
		\item $d(x,S) \leq d(x,y) + d(y,S)$ by the triangular inequality;
		\item $x \mapsto d(x,S)$ is a continuous function on $M$ by definition and the continuity of distance function.
	\end{enumerate}
	\begin{rmk}
	    Let $S$ be a closed set. First, clearly $d(\cdot,S)$ can be smooth around a neighborhood of $M \backslash S$. Besides, $(1)$ implies that $d(\cdot,S)$ is $1$-Lipschitz continuous, which means $d(\cdot,S)$ is continuously differentiable \emph{a.e.} on $M \backslash S$. Moreover. if $S$ is further a smooth submanifold, $d(\cdot,S)$ is smooth \emph{a.e.} on $M \backslash S$.
	\end{rmk}

	\begin{thm}
	    Let $(M,g)$ be a Riemannian manifold and $S \subset M$ be any subset and $f \colon M \sto [0,\infty)$ be $f(x) = d(x,S)$. If $f$ is $C^1$ on some open set $U \subset M \backslash S$, then $\abs{\op{grad} f} \equiv 1$ on $U$.
	\end{thm}
	\begin{proof}
	   Let $x \in U$. First, we want to prove $\abs{\op{grad} f(x)} \leq 1$. Assume $\abs{\op{grad} f} \neq 0$. Let $v \in T_xM$ with $\abs{v} = 1$ and $\gamma$ be the shortest geodesic with $\gamma(0) = x$ and $\dot{\gamma}(0) = v$. By above,
	   \begin{equation*}
	       f(\gamma(t)) \leq t + f(\gamma(0))~\Rightarrow~\left.\frac{d}{d t}\right|_{t=0} f(\gamma(t))=\lim _{t\sto 0} \frac{f(\gamma(t))-f(\gamma(0))}{t} \leq 1
	   \end{equation*}
	   On the other hand, if we let $v = (\op{grad}f(x)) / \abs{\op{grad}f(x)}$, then
	   \begin{equation*}
	       \left.\frac{d}{d t}\right|_{t=0} f(\gamma(t)) = \inn{\op{grad}f(x),v} = \abs{\op{grad}f(x)} \leq 1
	   \end{equation*}
	   Next, assume $\abs{\op{grad}f(x)} < 1$. By continuity of $\op{grad}f$, there exist $\delta,\varepsilon > 0$ such that $\abs{\op{grad} f} \leq 1 -\delta$ on a closed geodesic ball $\clo{B}_\varepsilon(x) \subset U$. Let $0<c<\varepsilon\delta$. There is arc length parametrized curve $\alpha \colon [0,b] \sto M$ such that
	   \begin{equation*}
	       \alpha(0)= x,~\alpha(b) \in S,~b = \op{Leng}(\alpha) < d(x,S) + c
	   \end{equation*}
	   Furthermore, if $b < \varepsilon$, $\alpha(b) \in \clo{B}_\varepsilon(x) \subset M \backslash S$, contradicting to $\alpha(b) \in S$, so $b > \varepsilon$. It follows that
	   \begin{equation*}
	       d(\alpha(\varepsilon),S) \leq \op{Leng}(\alpha|_{[\varepsilon,b]}) = b-\varepsilon < d(x,S) + c -\varepsilon
	   \end{equation*}
	   However, from another view, for $0 \leq t \leq \varepsilon$, $\alpha(t) \in \clo{B}_\varepsilon(x)$ implies that
	   \begin{equation*}
	       \abs{\frac{d}{t}f(\alpha(t))} = \abs{\inn{\op{grad}f,\dot{\alpha}(t)}} \leq \abs{\op{grad}f}\abs{\dot{\alpha}(t)} \leq 1-\delta ~\Rightarrow~\frac{d}{d t} f(\alpha(t)) \geq -(1-\delta)
	   \end{equation*}
	   which follows that
	   \begin{equation*}
	       f(\alpha(t)) \geq f(x)-(1-\delta) t
	   \end{equation*}
	   By setting $t = \varepsilon$,
	   \begin{equation*}
	       d(\alpha(\varepsilon), S) \geq d(x, S)-(1-\delta) \varepsilon >  d(x,S) + c -\varepsilon
	   \end{equation*}
	   which induces a contradiction.
	\end{proof}

	\begin{defn}
	    Let $(M,g)$ be a Riemannian manifold and $U \subset M$ be an open set. A local distance function on $U$ is a $C^1$ function $f \colon U \sto \R$ such that $\abs{\op{grad} f} \equiv 1$ on $U$.
	\end{defn}

	\begin{thm}
	    Suppose $f$ is a smooth local distance function on a open subset $U \subset M$. Then
	    \begin{equation*}
	        \nabla_{\op{grad}f}(\op{grad}f) \equiv 0
	    \end{equation*}
	    and so each integral curve of $\op{grad} f$ is a normal geodesic.
	\end{thm}
	\begin{proof}
	    Let $F = \op{grad} f$, \emph{i.e.} $W f=d f(W)=\langle F, W\rangle$ and so $F f=\langle F, F\rangle=|\operatorname{grad} f|^2=1$. Then we get
	    \begin{equation*}
	        \begin{aligned}
				\left\langle W, \nabla_F F\right\rangle & =F\langle W, F\rangle-\left\langle\nabla_F W, F\right\rangle \\
				& =F W f-\langle[F, W], F\rangle-\left\langle\nabla_W F, F\right\rangle \\
				& =F W f-[F, W] f-\frac{1}{2} W|F|^2 \\
				& =W F f-\frac{1}{2} W|F|^2 \\
				& =0
			\end{aligned}
	    \end{equation*}
	    Therefore, $\nabla_F F = 0$.
	\end{proof}

	\begin{lem}
	    Let $K \subset M$ and $f \colon K \sto \R$ is a continuous function and $f|_W$ is a smooth distance function on a open set $W \subset K$. Then for any curve $[a,b] \sto K$ with $\sigma \subset W$, we have
	    \begin{equation*}
	        \op{Leng}(\sigma) \geq \abs{f(\sigma(b)) - f(\sigma(a))}
	    \end{equation*}
	\end{lem}
	\begin{proof}
	    By $\abs{\op{grad} f} \equiv 1$,
	    \begin{equation*}
	        \begin{aligned}
	        	\abs{f(\sigma(b)) - f(\sigma(a))} &\leq \int_a^b \abs{\frac{d}{dt}f(\sigma(t))} dt \\
	        	&\leq \int_a^b \abs{\inn{\op{grad} f, \dot{\sigma(t)}}}dt \\
	        	&\leq \int_a^b \abs{\op{grad} f} \abs{\dot{\sigma(t)}} dt \\
	        	&= \int_a^b \abs{\dot{\sigma(t)}} dt = \op{Leng}(\sigma)\\
	        \end{aligned}
	    \end{equation*}
	    so we have the desired result.
	\end{proof}

	\begin{thm}
	    Let $(M,g)$ be a Riemannian manifold and $U \subset M$ be an open set. Let $S \subset U$ and $f \colon U \sto [0,\infty)$ be a continuous function such that $S = f^{-1}(0)$. Assume $f$ is a smooth local distance function on $U \backslash S$. Then there is a neighborhood $U_0 \subset U$ of $S$ in which $f(x) = d(x,S)$.
	\end{thm}
	\begin{proof}
	    Let
	    \begin{equation*}
	        U_0 = \bigcup_{p\in S}B_{\varepsilon_p}(p)
	    \end{equation*}
	    where $B_{\varepsilon_p}(p)$ is a $\delta_p$-uniform totally normal ball of $p$ and $\varepsilon_p$ is chosen such that $B_{w\varepsilon_p}(p) \subset U$. Let $x \in U_0$ arbitrary and $c = f(x)$. We need to check $d(x,S) = c$. First, if $x \in S$, it is clearly true. Assume $x \notin S$. There is some $p \in S$ such that $x \in B_{\varepsilon_p}(p)$, which means that $d(x,S) < \varepsilon_p$. Let $\alpha \colon [0,b] \sto B_{\varepsilon_p}(p)$ be the radical geodesic connecting $p$ to $x$. Then by above lemma,
	    \begin{equation*}
	        \op{Leng}(\alpha) \geq \abs{f(x)-f(p)} = c~\Rightarrow~c \leq \op{Leng}(\alpha) < \varepsilon_p
	    \end{equation*} 
	    Let $\gamma \colon (-\varepsilon_p,\varepsilon_p) \sto U$ be the normal geodesic starting from $x$ with $\dot{\gamma}(0) = -\op{grad}f(x)$. So
	    \begin{equation*}
	        \frac{d}{d t} f(\gamma(t))=\inn{\operatorname{grad} f, \dot{\gamma}(t)}=-\abs{\operatorname{grad} f}^2=-1
	    \end{equation*}
	    and
	    \begin{equation*}
	        f(\gamma(t)) = c - t
	    \end{equation*}
	    for $t < c <\varepsilon$, so $f(\gamma(c)) = 0$ by continuity, \emph{i.e.} $\gamma(c) \in S$. It follows that
	    \begin{equation*}
	        d(x,S) \leq c
	    \end{equation*} 
	    To prove the reverse inequality, suppose $\alpha:[a, b] \rightarrow M$ is any admissible curve starting at $x$ and ending at a point of $S$. Assume first that $\alpha(t) \in U$ for all $t \in[a, b]$, and let $b_0 \in[a, b]$ be the first time that $\alpha\left(b_0\right) \in S$. Then
	    \begin{equation*}
	        \op{Leng}(\alpha) \geq \op{Leng}\left(\left.\alpha\right|_{\left[a, b_0\right]}\right) \geq\left|f\left(\alpha\left(b_0\right)\right)-f(\alpha(a))\right|=c
	    \end{equation*}
	    On the other hand, suppose $\alpha(t) \in M \backslash U$ for some $t$. The triangle inequality implies, 
	    \begin{equation*}
	        B_{\varepsilon_p}(x) \subset B_{2 \varepsilon_p}(p) \subset U
	    \end{equation*}
	    so there is a first time $b_0 \in[a, b]$ such that $d\left(x, \alpha\left(b_0\right)\right) \geq \varepsilon_p$. Then
	    \begin{equation*}
	        \op{Leng}(\alpha) \geq \op{Leng}\left(\left.\alpha\right|_{\left[a, b_0\right]}\right) \geq \varepsilon_p>c
	    \end{equation*}
	    Taken together, these two inequalities show that $\op{Leng}(\alpha) \geq c$ for every such $\alpha$, which implies $d(x, S) \geq c$.
	\end{proof}
	\begin{cor}
	    Let $(M, g)$ be a Riemannian manifold, and let $f$ be a smooth local distance function on an open subset $U \subseteq M$. If c is a real number such that $S=f^{-1}(c)$ is nonempty, then there is a neighborhood $U_0$ of $S$ in $U$ on which $|f(x)-c|$ is equal to the distance in $M$ from $x$ to $S$.
	\end{cor}

	\item \emph{\textbf{Fermi Coordinates:}} Let $(M,g)$ be a Riemannian manifold and $P \subset M$ be an embedded submanifold. Let $\pi \colon NP \sto P$ be the normal bundle of $P$ in $m$.
	\begin{equation*}
	    \mathcal{E} \defeq \bb{(p,v) \in TM \colon \exp_p(tv) \text{ defined on }[0,1]}
	\end{equation*}
	called the domain of exponential map, that is $\exp \colon \mathcal{E} \sto M$ by $(p,v) \mapsto \exp_p(v)$. Then let $\mathcal{E}_P \defeq \mathcal{E} \cap NP$ and $E = \exp|_\mathcal{E} \colon \mathcal{E}_P \sto M$ be called normal exponential map of $P$ in $M$. 
	\begin{defn}
	    \begin{enumerate}[label=(\arabic{*})]
	    	\item A normal neighborhood of $P$ in $M$ is an open subset $U \subset M$ such that $E \colon V=E^{-1}(U) \sto U$ is diffeomorphic and $V \subset \mathcal{E}_P$ open, whose intersection with each fiber $N_xP$ is star-shaped.

	    	\item  A normal neighborhood of $P$ in $M$ is called a tubular neighborhood if 
	    	\begin{equation*}
	    	    V = \bb{(x,v) \in NP \colon \abs{v} < \delta(x)}
	    	\end{equation*}
	    	for some continuous function $\delta \colon P \sto (0,\infty)$. If $\delta(x) \equiv \varepsilon$, it is called a $\varepsilon$-uniform tubular neighborhood, or $\varepsilon$-tubular neighborhood.
	    \end{enumerate}
	\end{defn}

	\begin{thm}[Tubular Neighborhood Theorem]
	    Let $(M,g)$ be a Riemannian manifold. Every embedded submanifold $M$ has a tubular neighborhood in $M$ and every compact submanifold has a uniform tubular neighborhood.
	\end{thm}
	\begin{proof}
		Let $P \subset M$ be an embedded submanifold and $P_0 = \bb{(x,0) \colon x\in P} \subset NP$. 
		\begin{enumerate}[label=(\roman*)]
			\item First, we want to show there is a neighborhood of $P_0$ such that $E$ is a local diffeomorphism on it. It is sufficient to show $dE_{(x,0)}$ is bijective on each $(x,0) \in P_0$.

		    \noindent Note that $E|_{P_0} \colon P_0 \sto P$ is a diffeomorphism by $P \hookrightarrow M$, so
		    \begin{equation*}
		        dE_{(x,0)} \colon T_{(x,0)}P_0 \subset T_{(x,0)}NP \longrightarrow T_xP
		    \end{equation*}
		    is isomorphic. On the other hand, $E|_{N_xP} = \exp_x$, which is a local  diffeomorphism. $dE_{(x,0)} \colon T_{(x,0)}N_xP \sto N_xP$ is isomorphic. So
		    \begin{equation*}
		        dE_{(x,0)} \colon T_{(x,0)}NP \longrightarrow T_xM = T_xP \oplus N_xP
		    \end{equation*}
		    is surjective, so it is bijective by dimension equality. So $E$ is a local diffeomorphism. For any $(x,0) \in P_0$, there is a sufficiently small $\delta>0$ such that $E$ is diffeomorphic on
		    \begin{equation*}
		        V_\delta(x)=\left\{\left(x^{\prime}, v^{\prime}\right) \in N P: d\left(x, x^{\prime}\right)<\delta,\left|v^{\prime}\right|<\delta\right\}
		    \end{equation*}

		    \item For any $x \in P$, define
		    \begin{equation*}
		        \Delta(x) \defeq \sup\bb{\delta \leq 1 \colon E \text{ diffeomorphic on } V_{\delta}(x)}
		    \end{equation*}
		    By above, $\Delta(x) > 0$. Moreover, $E$ is injective on $V_{\Delta(x)}(x)$ because any $\left(x_1, v_1\right),\left(x_2, v_2\right)$ in this set are in $V_\delta(x)$ for some $\delta<\Delta(x)$. So $E$ is diffeomorphic on $V_{\Delta(x)}(x)$. Moreover, for any $x,x^\prime \in P$, if $d(x,x^\prime) < \Delta(x)$, by triangular inequality
		    \begin{equation*}
		       V_\delta(x^\prime) \subset V_{\Delta(x)}(x),\quad \delta = \Delta(x) - d(x,x^\prime)
		    \end{equation*}
		    This implies that $\Delta(x^\prime) \geq \Delta(x) - d(x,x^\prime)$, \emph{i.e.}
		    \begin{equation*}
		        \Delta(x)-\Delta\left(x^{\prime}\right) \leq d\left(x, x^{\prime}\right)
		    \end{equation*}
		    and it is also true when $d(x,x^\prime) > \Delta(x)$. Thus
		    \begin{equation*}
		        \left|\Delta(x)-\Delta\left(x^{\prime}\right)\right| \leq d\left(x, x^{\prime}\right)
		    \end{equation*}
		    which means $\Delta$ is continuous.

		    \item Consider
		    \begin{equation*}
		        V=\left\{(x, v) \in N P:|v|_g<\frac{1}{2} \Delta(x)\right\}
		    \end{equation*}
		    Let $(x,v),(x^\prime,v^\prime) \in V$ with $E(x,v)= E(x^\prime,v^\prime)$. Assume $\Delta(x^\prime) \leq \Delta(x)$. Because $\exp_x(v) = \exp_{x^\prime}(v^\prime)$, there is a curve from $x$ to $x^\prime$ with length $\abs{v}+\abs{v^\prime}$. So
		    \begin{equation*}
		        d\left(x, x^{\prime}\right) \leq|v|+\left|v^{\prime}\right|<\frac{1}{2} \Delta(x)+\frac{1}{2} \Delta\left(x^{\prime}\right) \leq \Delta(x)
		    \end{equation*}
		    Therefore, $(x,v),(x^\prime,v^\prime) \in V_{\Delta(x)}(x)$, so $(x,v)=(x^\prime,v^\prime)$. Because $E$ is a local diffeomorphism, $E \colon V \sto E(V)$ is a diffeomorphism. \qedhere
		\end{enumerate}
	\end{proof}
	Assume $\dim P = p$. Let $(W_0,\psi=(x^1,\cdots,x^p))$ be a coordinate chart of $P$ and $E_1,\cdots,E_{n-p}$ be a local orthogonal frame of $NP$. Let
	\begin{equation*}
	    V_0 = V \cap NP|_{W_0} \subset NP,\quad U_0 = E(V_0) \subset M
 	\end{equation*}
 	Then define the coordinate map $\varphi \colon U_0 \sto \R^n$ by
 	\begin{equation*}
 	    E(q,v^1E_1|_q+\cdots +v^{n-p}E_{n-p}|_q) \mapsto (x^1(q),\cdots,x^p(q),v^1,\cdots,v^{n-p})
 	\end{equation*}
 	called Fermi coordinates.

 	\begin{prop}
 	    Using notations as above and let $x^{p+j} = v^j$ for convenience.
 	    \begin{enumerate}[label=(\roman*)]
 	    	\item $q \in P \cap U_0$ has the coordinate $x^{p+1} = \cdots = x^n =0$.
 	    	\item At each $q \in P \cap U_0$,
 	    	\begin{equation*}
 	    	    g_{i j}=g_{j i}= \begin{cases}0, & 1 \leq i \leq p \text { and } p+1 \leq j \leq n, \\ \delta_{i j}, & p+1 \leq i, j \leq n .\end{cases}
 	    	\end{equation*}
 	    	\item For any $q \in P \cap U_0$ and $v=\left.v^1 E_1\right|_q+\cdots+\left.v^{n-p} E_{n-p}\right|_q \in N_q P$, the geodesic $\gamma$ starting from $\gamma(0) = q$ with initial velocity $v$ has coordinate expression
 	    	\begin{equation*}
 	    	    \gamma(t)=\left(x^1(q), \ldots, x^p(q), t v^1, \ldots, t v^{n-p}\right)
 	    	\end{equation*}
 	    	\item At each $q \in P \cap U_0$, $\Gamma^k_{ij}(q) = 0$ for $p+1 \leq i,j \leq n$.
 	    	\item At each $q \in P \cap U_0$,
 	    	\begin{equation*}
 	    	    \frac{\partial}{\partial x^i}g_{jk}(q) = 0,\quad p+1 \leq i,j,k \leq n
 	    	\end{equation*} 
 	    \end{enumerate}
 	\end{prop}

 	\item \emph{\textbf{Distance to Submanifolds:}} Let $(M,g)$ be a $n$-dimensional Riemannian manifold and $S \subset M$ be a $k$-dimensional submanifold. Let $U$ be any normal neighborhood of $S$ in $M$.
 	\begin{prop}
 	    There exist a unique continuous function $r \colon U \sto [0,\infty)$ and smooth vector field $V_r$ on $U \backslash S$ that has the coordinate representation in terms of any Fermi coordinates $(x^1,\cdots,x^k,v^1,\cdots,v^{n-k})$ for $S$ on a subset $U_0 \subset U$
 	    \begin{equation*}
 	        \begin{aligned}
				r\left(x^1, \ldots, x^k, v^1, \ldots, v^{n-k}\right) & =\sqrt{\left(v^1\right)^2+\cdots+\left(v^{n-k}\right)^2}, \\
				\partial_r & =\frac{v^1}{r(x, v)} \frac{\partial}{\partial v^1}+\cdots+\frac{v^{n-k}}{r(x, v)} \frac{\partial}{\partial v^{n-k}} .
			\end{aligned}
 	    \end{equation*}
 	    Moreover, $r$ is smooth on $U \backslash S$, and $r^2$ is smooth on all of $S$.
 	\end{prop}
 	\begin{proof}
 	    It only needs to prove $r$ and $V_r$ can be defined on $U$ and $U \backslash S$ respectively. Let $E \colon V \subset NS \sto U$ be diffeomorphic.
 	    \begin{enumerate}[label=(\roman*)]
 	    	\item Define a function $\rho \colon V \sto [0,\infty)$ by $\rho(p,v) = \abs{v}$ and $r \colon U \sto [0,\infty)$ by $r = \rho \circ E^{-1}$.
 	    	\item Let $q \in U \backslash S$. Then $q = \exp_p(v)$ for a unique $(p,v) \in V$, and the geodesic $\gamma \colon [0,1] \sto U$ given by $\gamma(t) = \exp_p(tv)$ connecting $p$ to $q$, which has the coordinate expression
 	    	\begin{equation*}
 	    	    \gamma(t) = (x^1,\cdots,x^k,tv^1,\cdots,tv^{n-k}),\quad x(q)= (x^i),~v = v^iE_i|_{p}
 	    	\end{equation*}
 	    	Define
 	    	\begin{equation*}
 	    	    V_r(q) \defeq \frac{1}{r(q)}\dot{\gamma}(1) \qedhere
 	    	\end{equation*}
 	    \end{enumerate}
 	\end{proof}

 	\begin{thm}[Gauss Lemma]
 	    With the same notations, on $U \backslash S$, $V_r$ is a unit vector field orthogonal to the level sets of $r$.
 	\end{thm}
 	\begin{proof}
 		\begin{enumerate}[label=(\roman*)]
 			\item For $q \in U \backslash S$ with coordinate $(x^1,\cdots,x^k,v^1,\cdots,v^{n-k})$, consider the corresponding geodesic
	 	    \begin{equation*}
	 	        \gamma(t) = \exp_p(tv) = (x^1,\cdots,x^k,tv^1,\cdots,tv^{n-k})
	 	    \end{equation*}
	 	    Then
	 	    \begin{equation*}
	 	        \left|\gamma^{\prime}(0)\right|=|v|=\sqrt{\left(v^1\right)^2+\cdots+\left(v^{n-k}\right)^2}=r(q)
	 	    \end{equation*}
	 	    and thus $\abs{\dot{\gamma}(1)} = r(q)$, and $\abs{V_r(q)} = 1$. 

	 	    \item  For orthogonality, let $q \in U \backslash S$ and $q = \exp_{p_0}(v_0)$ for $p_0 \in S$ and $v_0 \in N_{p_0}S$ with $v_0 \neq 0$. Let $b = r(q) = \abs{v_0}$. So $q \in r^{-1}(b)$, which is an embedded submanifold. Let $\sigma \colon (-\varepsilon,\varepsilon) \sto r^{-1}(q)$ be starting from $q$ and denote $w = \dot{\sigma}(0)$. Then
	 	    \begin{equation*}
	 	        \sigma(s) = \exp_{x(s)}(v(s))
	 	    \end{equation*}
	 	    where $x(s) \in S$ with $x(0) = p_0$, $v(s) \in N_{x(s)}S$ with $\abs{v(s)} \equiv b$ and $v(0)=v_0$. Consider a variation
	 	    \begin{equation*}
	 	        F(s,t) = \exp_{x(s)}\bc{\frac{t}{b}v(s)} \colon (-\varepsilon,\varepsilon) \times [0,b] \sto M
	 	    \end{equation*}
	 	    with $T(t) = \frac{\partial}{\partial t}F(0,t)$ and Jacobian field $J(t) = \frac{\partial}{\partial s}F(0,t)$.
	 	    \begin{equation*}
	 	        T(0) = v_0,\quad T(b) = V_r(q)
	 	    \end{equation*}
	 	    and
	 	    \begin{equation*}
	 	        J(0) = \dot{x}(0),\quad J(b) = w
	 	    \end{equation*}
	 	    Consider the energy
	 	    \begin{equation*}
	 	        E(s) = \frac{1}{2} \int_0^b \inn{\frac{\partial}{\partial t}F(s,t),\frac{\partial}{\partial t}F(s,t)}dt = \frac{1}{2} \int_0^b \abs{v(s)}^2dt = \frac{1}{2}r^2b
	 	    \end{equation*}
	 	    we have
	 	    \begin{equation*}
	 	        \begin{aligned}
	 	            0 = \dot{E}(0) &= \inn{J(b),T(b)} - \inn{J(0),T(0)} \\
	 	            &= \inn{w,V_r(q)}
	 	        \end{aligned}
	 	    \end{equation*}
	 	    because $\inn{\dot{x}(0),v_0} = 0$. \qedhere
 		\end{enumerate}
 	\end{proof}
 	\begin{cor}
 	    Using same notations and assumptions as above,
 	    \begin{enumerate}[label=(\arabic{*})]
 	    	\item $V_r = \op{grad} r$ on $U \backslash S$.
 	    	\item $r$ is a local distance function
 	    	\item  each unit-speed geodesic $\gamma:[a, b) \rightarrow U$ with $\dot{\gamma}(a)$ normal to $S$ coincides with an integral curve of $V_r$ on $(a, b)$.
			\item  $S$ has a tubular neighborhood in which the distance in $M$ to $S$ is equal to $r$.
 	    \end{enumerate}
 	\end{cor}
\end{enumerate}

\section{Riemannian Submanifold}

Let $(M,g,\nabla)$ be a Riemannian (embedded) submanifold of $(\widetilde{M},\widetilde{g},\widetilde{\nabla})$, i.e., $ \iota^*g = \widetilde{g}$ for $\iota: M \hookrightarrow \widetilde{M}$. Define
\begin{align*}
	\pi^\top &\colon T\widetilde{M}|_M \sto TM, \\
	\pi^\perp &\colon T\widetilde{M}|_M \sto NM,
\end{align*}
be the tangential and normal projection. Denote $X^\top = \pi^\top(X)$, $X^\perp = \pi^{\perp}(X)$ for any $X \in T\widetilde{M}$. Also, for any $X,Y \in \Gamma(TM)$, they can be extended to vector fields on an open set of $\widetilde{M}$, and also denoted by $X,Y$. Then
\begin{equation*}
    \widetilde{\nabla}_XY = (\widetilde{\nabla}_XY)^\top +(\widetilde{\nabla}_XY)^\perp
\end{equation*}

\begin{defn}[Second Fundamental Form]
    Using notations as above, the second fundamental form on $M$ is a map
    \begin{equation*}
        \mathrm{II} \colon \Gamma(TM) \times \Gamma(TM )\sto \Gamma(NM)
    \end{equation*}
    defined as
    \begin{equation*}
        \mathrm{II}(X, Y)=\left(\widetilde{\nabla}_X Y\right)^{\perp}.
    \end{equation*}
\end{defn}

\begin{prop}
    Using same notations as above.
    \begin{enumerate}[label=(\arabic{*})]
    	\item $\mathrm{II}(X,Y)$ is independent of the extensions of $X,Y$ to an open subset of $\widetilde{M}$.
    	\item $\mathrm{II}(X,Y)$ is bilinear over $C^\infty(M)$ in $X$ and $Y$.
    	\item $\mathrm{II}(X,Y)$ is symmetric.
    	\item $\mathrm{II}(X,Y)_p$ only dependents on $X_p$ and $Y_p$ for any $p \in M$.
    \end{enumerate}
\end{prop}
\begin{proof}
    For the symmetry, first,
    \begin{equation*}
        \mathrm{II}(X, Y)-\mathrm{II}(Y, X)=\left(\widetilde{\nabla}_X Y-\widetilde{\nabla}_Y X\right)^{\perp}=[X, Y]^{\perp}.
    \end{equation*}
    Because $X,Y \in \Gamma(TM)$, $[X,Y] \in \Gamma(TM)$ and so $[X,Y]^\perp = 0$. For the bilinearity over $C^\infty(M)$, it is because any $f \in C^\infty(M)$ can be extended to $\widetilde{f} \in C^\infty(\widetilde{M})$ and by the symmetry. All independence is because of the properties of connection.
\end{proof}
\begin{rmk}
    By the independence, it can safely define $\mathrm{II}(v,w)$ for $v,w \in T_pM$.
\end{rmk}

\begin{thm}[Gauss Formula]
    Using notations as above, for $X,Y \in \Gamma(TM)$ and extended on an open set on $\widetilde{M}$, then on $M$, we have
    \begin{equation*}
        \widetilde{\nabla}_XY = \nabla_XY + \mathrm{II}(X,Y),
    \end{equation*}
    i.e., $\nabla_XY = (\widetilde{\nabla}_XY)^\top$.
\end{thm}
\begin{proof}
    Define $\nabla^\top \colon \Gamma(TM) \times \Gamma(TM) \sto \Gamma(TM)$ as
    \begin{equation*}
        \nabla^\top_XY = (\widetilde{\nabla}_XY)^\top.
    \end{equation*}
    It suffices to show $\nabla^\top$ is a Levi-Civita connection.

    First, because $\widetilde{\nabla}$ is an affine connection and $\pi^\top$ is linear, $\nabla^\top$ is obviously an affine connection on $M$. For the consistency with metric,
    \begin{align*}
		\nabla_X^{\top}\langle Y, Z\rangle & =X\langle Y, Z\rangle= X \langle Y ,  Z \rangle \\
		& =\widetilde{\nabla}_{ X }\langle Y ,  Z \rangle \\
		& =\left\langle\widetilde{\nabla}_{ X }  Y ,  Z \right\rangle+\left\langle Y , \widetilde{\nabla}_{ X }  Z \right\rangle \\
		& =\left\langle\pi^{\top}\left(\widetilde{\nabla}_{ X }  Y \right),  Z \right\rangle+\left\langle Y , \pi^{\top}\left(\widetilde{\nabla}_{ X }  Z \right)\right\rangle \\
		& =\left\langle\nabla_X^{\top} Y, Z\right\rangle+\left\langle Y, \nabla_X^{\top} Z\right\rangle.
    \end{align*}
    For the torsion free,
    \begin{align*}
		\nabla_X^{\top} Y-\nabla_Y^{\top} X & =\pi^{\top}\left(\left.\widetilde{\nabla}_{ X }  Y \right|_M-\left.\widetilde{\nabla}_{ Y }  X \right|_M\right) \\
		& =\pi^{\top}\left(\left.[ X ,  Y ]\right|_M\right) \\
		& =\left.[ X ,  Y ]\right|_M \\
		& =[X, Y]. \qedhere
    \end{align*}
\end{proof}

For each normal vector field $N \in \Gamma(NM)$, define
\begin{equation*}
    \mathrm{II}_N \colon \Gamma(TM) \times \Gamma(TM) \sto C^\infty(M)
\end{equation*}
as
\begin{equation*}
    \mathrm{II}_N(X,Y) = \inn{N,\mathrm{II}(X,Y)}.
\end{equation*}
So $\mathrm{II}_N \in \Gamma(\otimes^{0,2}TM)$ and is symmetric. Let $W_N \colon \Gamma(TM) \sto \Gamma(TM)$ be a self-adjoint linear operator defined as
\begin{equation*}
    \inn{W_N(X),Y} = \mathrm{II}_N(X,Y) = \inn{N,\mathrm{II}(X,Y)}.
\end{equation*}
Then $W_N$ is called the Weingarten map in the direction of $N$. Because $\mathrm{II}$ is bilinear over $C^\infty(M)$, $W_N$ is $C^\infty(M)$-linear, which implies that $W_N \colon TM \sto TM$.

\begin{prop}[Weingarten Equation]
    Using notations as above, for any $X \in \Gamma(TM)$ and $N \in \Gamma(NM)$,
    \begin{equation*}
        (\widetilde{\nabla}_XN)^\top = - W_N(X),
    \end{equation*}
    where $N$ is extended arbitrarily to an open subset of $\widetilde{M}$.
\end{prop}
\begin{proof}
    Let $Y \in \Gamma(TM)$ and be extended arbitrarily on an open set of $\widetilde{M}$. Because $\inn{N,Y} = 0$ on $M$,
    \begin{align*}
    	0&=X\inn{N,Y} \\
    	&=\inn{\widetilde{\nabla}_XN,Y} + \inn{N,\widetilde{\nabla}_XY} \\
    	&= \inn{\widetilde{\nabla}_XN,Y} + \inn{N,\nabla_XY+\mathrm{II}(X,Y)} \\
    	&=\inn{\widetilde{\nabla}_XN,Y} + \inn{N,\mathrm{II}(X,Y)} \\
    	&=\inn{\widetilde{\nabla}_XN,Y} + \inn{W_N(X),Y} \\
    	&= \inn{(\widetilde{\nabla}_XN)^\top + W_N(X),Y}. \qedhere
    \end{align*}
\end{proof}

\begin{thm}[Gauss Equation]
    Let $R$ and $\widetilde{R}$ be the corresponding Riemannian curvature of $M$ and $\widetilde{M}$ respectively. For any $X,Y,Z,W \in \Gamma(TM)$,
    \begin{equation*}
        \widetilde{R}(Z,Y,W,X) = R(Z,Y,W,X) - \inn{\mathrm{II}(W,Z),\mathrm{II}(X,Y)} + \inn{\mathrm{II}(W,Y),\mathrm{II}(X,Z)}.
    \end{equation*}
\end{thm}
\begin{proof}
    By definition,
    \begin{align*}
    	\widetilde{R}(Z,Y,W,X) &= \inn{\widetilde{\nabla}_W \widetilde{\nabla}_X Y-\widetilde{\nabla}_X \widetilde{\nabla}_W Y-\widetilde{\nabla}_{[W, X]} Y, Z} \\
    	&=\inn{\widetilde{\nabla}_W (\nabla_XY+\mathrm{II}(X,Y))-\widetilde{\nabla}_X (\nabla_WY+\mathrm{II}(W,Y))-\widetilde{\nabla}_{[W, X]} Y, Z}.
    \end{align*}
    Note that because $\mathrm{II} \in \Gamma(NM)$, by Weingarten equation,
    \begin{equation*}
        (\widetilde{\nabla}_W\mathrm{II}(X,Y))^\top = W_{\mathrm{II}(X,Y)}(W),\quad (\widetilde{\nabla}_X\mathrm{II}(W,Y))^\top = W_{\mathrm{II}(W,Y)}(X).
    \end{equation*}
    Therefore, by $Z \in \Gamma(TM)$,
    \begin{align*}
    	\widetilde{R}(Z,Y,W,X) &= \left\langle\widetilde{\nabla}_W \nabla_X Y, Z\right\rangle-\langle\mathrm{II}(X, Y), \mathrm{II}(W, Z)\rangle \\
		& -\left\langle\widetilde{\nabla}_X \nabla_W Y, Z\right\rangle+\langle\mathrm{II}(W, Y), \mathrm{II}(X, Z)\rangle-\left\langle\widetilde{\nabla}_{[W, X]} Y, Z\right\rangle \\
		&=\left\langle{\nabla}_W \nabla_X Y, Z\right\rangle-\langle\mathrm{II}(X, Y), \mathrm{II}(W, Z)\rangle \\
		& -\left\langle{\nabla}_X \nabla_W Y, Z\right\rangle+\langle\mathrm{II}(W, Y), \mathrm{II}(X, Z)\rangle-\left\langle{\nabla}_{[W, X]} Y, Z\right\rangle \\
		&=R(Z,Y,W,X) - \inn{\mathrm{II}(W,Z),\mathrm{II}(X,Y)} + \inn{\mathrm{II}(W,Y),\mathrm{II}(X,Z)}. \qedhere
    \end{align*}
\end{proof}

Similarly, it can define the normal connection,
\begin{equation*}
    \nabla^\perp \colon \Gamma(TM) \times \Gamma(NM) \sto \Gamma(NM)
\end{equation*}
is defined as
\begin{equation*}
    \nabla^\perp_XN = (\widetilde{\nabla}_XN)^\perp.
\end{equation*}
Then $\nabla^\perp$ is a well-defined affine connection defined on the normal bundle $NM$, which coincides with the metric, i.e.,
\begin{equation*}
    X\left\langle N_1, N_2\right\rangle=\left\langle\nabla_X^{\perp} N_1, N_2\right\rangle+\left\langle N_1, \nabla_X^{\perp} N_2\right\rangle .
\end{equation*}

More generally, let $\pi \colon F \sto M$ be a vector bundle whose fiber
\begin{equation*}
    F_p = \bb{ B_p \colon T_pM \times T_pM \sto N_pM \colon B_p \text{ is bilinear.} },
\end{equation*}
i.e., $F = T^*M \otimes T^*M \otimes NM$. Note any section $B \in \Gamma(F)$ is
\begin{equation*}
    B \colon \Gamma(TM) \times \Gamma(TM) \sto \Gamma(NM)
\end{equation*}
that is bilinear over $C^\infty(M)$. Define a connection $\nabla^F$ on $F$
\begin{equation*}
    \nabla^F \colon \Gamma(TM) \times \Gamma(F) \sto \Gamma(F),
\end{equation*}
as
\begin{equation*}
    (\nabla^F_XB)(Y,Z) = \nabla^\perp_X(B(Y,Z)) - B(\nabla_XY,Z) - B(Y,\nabla_XZ).
\end{equation*}
Note that $\nabla^F_XB$ is true symmetric and $\nabla^F$ is an affine connection.

\begin{thm}[Codazzi Equation]
    Let $(M,g) \subset (\widetilde{M},\widetilde{g})$ be an embedded Riemannian manifold. For all $W,X,Y \in \Gamma(TM)$,
    \begin{equation*}
        (\widetilde{R}(W,X)Y)^\perp = (\nabla^F_W \mathrm{II})(X,Y) - (\nabla^F_X\mathrm{II})(W,Y).
    \end{equation*}
\end{thm}
\begin{proof}
    It suffices to prove that for any $N \in \Gamma(NM)$,
    \begin{equation*}
        \langle\tilde{R}(W, X) Y, N\rangle=\left\langle\left(\nabla_W^F \mathrm{II}\right)(X, Y), N\right\rangle-\left\langle\left(\nabla_X^F \mathrm{II}\right)(W, Y), N\right\rangle.
    \end{equation*}
    For the LHS,
    \begin{equation*}
    	\tilde{R}(N,Y,W,X) = \inn{\widetilde{\nabla}_W\left(\nabla_X Y+\mathrm{II}(X, Y)\right)-\widetilde{\nabla}_X\left(\nabla_W Y+\mathrm{II}(W, Y)\right) -\tilde{\nabla}_{[W, X]} Y, N}.
    \end{equation*}
    Note that
    \begin{align*}
        (\widetilde{\nabla}_W\mathrm{II}(X, Y))^\perp &= \nabla^\perp_W \mathrm{II}(X, Y) \\
        &= (\nabla^F_W \mathrm{II})(X, Y) + \mathrm{II}(\nabla_WX, Y) + \mathrm{II}(X, \nabla_WY),\\
        (\widetilde{\nabla}_X\mathrm{II}(W, Y))^\perp &= \nabla^\perp_X \mathrm{II}(W, Y) \\
        &= (\nabla^F_X \mathrm{II})(W, Y) + \mathrm{II}(\nabla_XW, Y) + \mathrm{II}(W, \nabla_XY),
    \end{align*}
    and
    \begin{align*}
        (\widetilde{\nabla}_W\nabla_X Y)^\perp &= \mathrm{II}(\nabla_X Y,W).\\
        (\widetilde{\nabla}_X\nabla_W Y)^\perp &= \mathrm{II}(\nabla_W Y,X),\\
        (\tilde{\nabla}_{[W, X]} Y)^\perp &= \mathrm{II}([W,X],Y).
    \end{align*}
    Therefore,
    \begin{align*}
    	\tilde{R}(N,Y,W,X) &= \left\langle\mathrm{II}\left(W, \nabla_X Y\right)+\left(\nabla_W^F \mathrm{II}\right)(X, Y)+\mathrm{II}\left(\nabla_W X, Y\right)+\mathrm{II}\left(X, \nabla_W Y\right), N\right\rangle \\
		& -\left\langle\mathrm{II}\left(X, \nabla_W Y\right)+\left(\nabla_X^F \mathrm{II}\right)(W, Y)+\mathrm{II}\left(\nabla_X W, Y\right)+\mathrm{II}\left(W, \nabla_X Y\right), N\right\rangle \\
		& -\langle\mathrm{II}([W, X], Y), N\rangle\\
		&= \left\langle\left(\nabla_W^F \mathrm{II}\right)(X, Y), N\right\rangle-\left\langle\left(\nabla_X^F \mathrm{II}\right)(W, Y), N\right\rangle.\qedhere
    \end{align*}
\end{proof}

\begin{exam}
    Let $\gamma \colon I \sto M$ be a normal curve. The geodesic curvature of $\gamma$ in $M$
    \begin{equation*}
        \kappa(t) = \abs{\frac{D}{dt}\dot{\gamma}(t)}
    \end{equation*}
    If $(M,g) \subset (\widetilde{M},\widetilde{g})$ is an embedded Riemannian manifold, then
    \begin{equation*}
        \frac{\widetilde{D}}{dt}\dot{\gamma}(t) = \frac{D}{dt}\dot{\gamma}(t) + \mathrm{II}(\dot{\gamma}(t),\dot{\gamma}(t)).
    \end{equation*}
    Therefore, for any $v \in T_pM$, $\abs{\mathrm{II}(v,v)}$ is the geodesic curvature of the normal geodesic $\gamma$ in $M$ with $\dot{\gamma}(0) = v$.
\end{exam}

\section{Hypersurface}

Let $M \subset \widetilde{M}$ be a Riemannian hypersurface with $\dim M = n$ and $\dim \widetilde{M} = n+1$. In such case, at each point, there exactly two unit normal vectors of $M$. Moreover, we assume a smooth unit normal field $N$ exists on an enough big neighborhood.

Consider the scalar second fundamental form $h \in \Gamma(\otimes^{(0,2)}TM)$, defined as
\begin{equation*}
    h(X,Y) = \mathrm{II}_N(X,Y) = \inn{N,\mathrm{II}(X,Y)} = \inn{N,\widetilde{\nabla}_XY}.
\end{equation*}
Because $N$ is the unit vector spanning $NM$,
\begin{equation*}
    \mathrm{II}(X,Y) = h(X,Y)N.
\end{equation*}
We use notation $s = W_N \colon \Gamma(TM) \sto \Gamma(TM)$ a Weingarten map, called the shape operator of $M$, so
\begin{equation*}
    \inn{sX,Y} = h(X,Y),\quad X,Y \in \Gamma(TM).
\end{equation*}
Moreover, the symmetry of $h$ implies that
\begin{equation*}
    \inn{sX,Y} = \inn{X,sY},\quad X,Y \in \Gamma(TM),
\end{equation*}
so $s$ is self-adjoint, which implies that $s$ is diagonalizable by orthonormal basis $b_1,\cdots,b_n$ that are eigenvectors w.s.t. eigenvalues $\kappa_1,\cdots,\kappa_n$. Then $\kappa_i$ are called principal curvatures with corresponding principal directions $b_i$. The Gaussian curvature is
\begin{equation*}
    K = \det(s) = \kappa_1\kappa_2\cdots \kappa_n.
\end{equation*}
So the mean curvature
\begin{equation*}
    H = \frac{1}{n}\tr(s) = \frac{1}{n}\tr_g(h) = \frac{1}{n}(\kappa_1+\kappa_2+\cdots+\kappa_n).
\end{equation*}

Give two notations:
\begin{enumerate}[label=\Roman{*}.]
	\item For two symmetric $(0,2)$-tensors $h,k$,
	\begin{align*}
	    h \odot k(w, x, y, z) &=h(w, z) k(x, y)+h(x, y) k(w, z) \\
	    &\quad -h(w, y) k(x, z)-h(x, z) k(w, y).
	\end{align*}

	\item For a smooth $(0,2)$-tensor field $T$ is
	\begin{equation*}
	    (DT)(x,y,z) = -(\nabla T)(x,y,z) + (\nabla T)(x,z,y).
	\end{equation*}
\end{enumerate}
\begin{thm}
    Let $M \subset \widetilde{M}$ be a Riemannian hypersurface and $N$ be a smooth unit normal vector field along $M$.
    \begin{enumerate}[label=(\arabic{*})]
    	\item Gauss formula: for $X,Y \in \Gamma(TM)$, then
    	\begin{equation*}
    	    \widetilde{\nabla}_XY = \nabla_XY + h(X,Y)N.
    	\end{equation*}

    	\item Gauss equation: for all $W,X,Y,Z \in \Gamma(TM)$,
    	\begin{equation*}
    	    \widetilde{R}(Z, Y, W, X)=R(Z, Y, W, X) - \frac{1}{2}h \odot h(W,X,Y,Z).
    	\end{equation*}

    	\item Weingarten equation: for $X \in \Gamma(TM)$, $\widetilde{\nabla}_XN \in \Gamma(TM)$ and
    	\begin{equation*}
    	    \widetilde{\nabla}_XN = -sX.
    	\end{equation*}

    	\item Codazzi equation: $W,X,Y,Z \in \Gamma(TM)$,
    	\begin{equation*}
    	    \widetilde{R}(N, Y, W, X) = (Dh)(Y,W,X)
    	\end{equation*}
    \end{enumerate}
\end{thm}
\begin{proof}
    $(1)$ and $(2)$ are obviously by above. For $(3)$, because
    \begin{equation*}
        \left\langle\widetilde{\nabla}_X N, N\right\rangle=\frac{1}{2} X\left(|N|^2\right)=0,
    \end{equation*}
    and $N$ span $NM$, $\widetilde{\nabla}_XN \in \Gamma(TM)$.

    For $(4)$, by above, $\nabla^\perp_XN = (\widetilde{\nabla}_X N)^\perp = 0$. So
    \begin{align*}
		\left(\nabla_W^F \mathrm{II}\right)(X, Y) & =\nabla_W^{\perp}(\mathrm{II}(X, Y))-\mathrm{II}\left(\nabla_W X, Y\right)-\mathrm{II}\left(X, \nabla_W Y\right) \\
		& =\nabla_W^{\perp}(h(X, Y) N)-\mathrm{II}\left(\nabla_W X, Y\right)-\mathrm{II}\left(X, \nabla_W Y\right) \\
		& =\left(W(h(X, Y))-h\left(\nabla_W X, Y\right)-h\left(X, \nabla_W Y\right)\right) N \\
		& =\nabla_W(h)(X, Y) N = (\nabla h)(Y,X,W)N.
    \end{align*}
    Then by above theorem,
    \begin{equation*}
        \widetilde{R}(N, Y, W, X) = (Dh)(Y,W,X). \qedhere
    \end{equation*}
\end{proof}

\paragraph{Calculation.} By using the tubular neighborhood of $M$ in $\widetilde{M}$, choose Fermi coordinate system
\begin{equation*}
    (U,x^1,x^2,\cdots,x^n,v),
\end{equation*}
which is called semigeodesic coordinates in the sense that
\begin{equation*}
    \gamma \colon t \mapsto (x^1,\cdots,x^{n-1},t)
\end{equation*}
is a normal geodesic and $\dot{\gamma}(t_0)$ is orthogonal to the level set $r(M) = t_0$ for the distance function $r \colon U \sto [0,\infty)$ of $M$ defined by $r(x^1,x^2,\cdots,x^n,v) = \abs{v}$. Using this coordinates, we have
\begin{equation*}
    \widetilde{g} = dv \otimes dv + g_{ij}(x^1,\cdots,x^n,v) dx^i \otimes dx^j,
\end{equation*}
by the fact that
\begin{equation*}
    \inn{\frac{\partial}{\partial v},\frac{\partial}{\partial v}} =1,\quad \inn{\frac{\partial}{\partial v},\frac{\partial}{\partial x^i}} = 0.
\end{equation*}
Moreover,
\begin{align*}
\widetilde{\Gamma}_{v v}^v&=\widetilde{\Gamma}_{v v}^i=\widetilde{\Gamma}_{i v}^v=\widetilde{\Gamma}_{v i}^v=0, \\
\widetilde{\Gamma}_{ij}^v&=-\frac{1}{2} \frac{\partial}{\partial v} g_{ij}, \\
\widetilde{\Gamma}_{vi}^j&=\widetilde{\Gamma}_{i v}^j=\frac{1}{2} g^{j k} \frac{\partial}{\partial v} g_{ki}, \\
\widetilde{\Gamma}_{ij}^k&=\Gamma_{ij}^k,
\end{align*}
Using this coordinate system, by choosing $N = \frac{\partial}{\partial v}$, consider the hypersurface $M_a = r^{-1}(a)$, the scalar second fundamental $h_a$, the shape operator $s_a$, and the mean curvature can be expressed as
\begin{align*}
	(h_a)_{ij} &= -\frac{1}{2} \lv{\frac{\partial}{\partial v}}_{v=a} g_{ij},\\
	(s_a)^i_j &= -\frac{1}{2}g^{ik}\lv{\frac{\partial}{\partial v}}_{v=a}g_{kj},\\
	H_a &= -\frac{1}{2n}g^{ij}\lv{\frac{\partial}{\partial v}}_{v=a}g_{ji}.
\end{align*}
