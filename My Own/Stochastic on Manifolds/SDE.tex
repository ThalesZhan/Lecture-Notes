\chapter{SDE and Diffusion}

Fix probability space $(\Omega, \mathcal{F}, \Pb)$ equipped with a filtration $\mathbb{F} = (\mathcal{F}_t)_{t \geq 0}$ of $\sigma$-fields contained in $\mathcal{F}$ with assumption that $\mathcal{F}_\infty = \mathcal{F}$. Also, $\mathbb{F}$ satisfies the usual condition, i.e., complete and right-continuous. Without specification, all semimartingales are assumed continuous. We use Einstein summation convention for convenience.

\section{SDE on Euclidean Space}

\paragraph{Solution of SDE:} Let $Z = (Z_t)_{t \geq 0}$ be a continuous $\R^\ell$-valued $\mathbb{F}$-semimartingale, i.e.,
\begin{equation*}
	Z = M + A,
\end{equation*}
where $M$ is a local martingale and $A$ is an adapted process of local bounded variation such that $A_0 = 0$. Let $\sigma \colon \R^N \sto \R^{N \times \ell}$ be smooth and locally Lipschitz, i.e. for all $R > 0$ there exists $R > 0$ and $C(R) > 0$ such that
\begin{equation*}
	\norm{\sigma(x) - \sigma(y)}_{\op{F}} \leq C(R)\norm{x - y},\quad \forall~x,y \in B(R).
\end{equation*}
Let $X_0 \in \R^N$ be $\mathcal{F}_0$ measurable. For a stopping time $\tau$, consider a semimartingale $X = (X_{t \wedge \tau})_{t \geq 0}$ of the form
\begin{equation*}
 	X_t = X_0 + \int_0^t \sigma(X_s) dZ_s,\quad 0 \leq t \leq \tau.
\end{equation*}
We say $X$ is a solution of a SDE driving by the semimartingale $Z$, and such equation denoted as $\op{SDE}(\sigma,Z,X_0)$.  Moreover, by It\^o formula, for $f \in C^2(\R^N)$,
\begin{align*}
	f(X_t)& =f\left(X_0\right)+\int_0^t f_{x_i}\left(X_s\right) \sigma_\alpha^i\left(X_s\right) d Z_s^\alpha \\
	& \quad+\frac{1}{2} \int_0^t f_{x_i x_j}\left(X_s\right) \sigma_\alpha^i\left(X_s\right) \sigma_\beta^j\left(X_s\right) d\left\langle Z^\alpha, Z^\beta\right\rangle_s \\
	& =f\left(X_0\right)+\int_0^t f_{x_i}\left(X_s\right) \sigma_\alpha^i\left(X_s\right) d M_s^\alpha+\int_0^t f_{x_i}\left(X_s\right) \sigma_\alpha^i\left(X_s\right) d A_s^\alpha \\
	& \quad+\frac{1}{2} \int_0^t f_{x_i x_j}\left(X_s\right) \sigma_\alpha^i\left(X_s\right) \sigma_\beta^j\left(X_s\right) d\left\langle M^\alpha, M^\beta\right\rangle_s
\end{align*}
\begin{rmk}
    Note that if $X_0 \in L^2$ and $\sigma$ is globally Lipschitz continuous, then we don't need the stopping time $\tau$ and $\op{SDE}(\sigma,Z,X_0)$ has a unique solution $X = (X_t)_{t \geq 0}$.
\end{rmk}

The problem why we need to consider a stopping time is because of the local Lipschitz property of $\sigma$. The local Lipschitz would make the usual solution exploded. To consider it rigorously, we first need the definition of explosion time.

Let $M$ be a locally compact metric space and $\widehat{M} = M \cup \bb{\partial_M}$ be its one-point compactification. A continuous map $x \colon [0,\infty) \sto \widehat{M}$ has an explosion time $e = e(x) > 0$ if $x_t \in M$ for $0 \leq t < e$ and $x_t = \partial M$ for $t \geq e$ when $e < \infty$, i.e.
\begin{equation*}
	e \colon W(M) \sto [0,\infty]
\end{equation*}
where $W(M) \subset C([0,\infty),\widehat{M})$ such that $e$ can be defined. Moreover, $W(M)$ can be equipped with the canonical filtration $\mathbb{F}_M = (\mathcal{F}_{M,t})_{t \geq 0}$, i.e., the natural filtration generated by coordinate processes. Then $e$ is a $\mathbb{F}_M$-stopping time.

\begin{defn}
    Let $\tau$ be a stopping time. A continuous $X = (X_{t \wedge \tau})_{t \geq 0}$ is called a semimartingale up to $\tau$ if there exists a sequence of stopping times $\tau_n \uparrow \tau$ such that $X^{\tau_n} = (X_{t \wedge \tau_n})$ is a semimartingale.
\end{defn}

\begin{defn}
    A semimartingale $X$ up to a stopping time $\tau$ is a solution of $\op{SDE}(\sigma,Z,X_0)$ if there exists a sequence of stopping times $\tau_n \uparrow \tau$ such that $X^{\tau_n} = (X_{t \wedge \tau_n})$ is a semimartingale and
    \begin{equation*}
     	X_{t \wedge \tau_n}=X_0+\int_0^{t \wedge \tau_n} \sigma\left(X_s\right) d Z_s, \quad t \geq 0.
    \end{equation*} 
\end{defn}

\begin{thm}
    If $\sigma$ is locally Lipschitz continuous, then there exists a unique $W(\R^N)$-valued random variable $X$ which is a solution of $\op{SDE}(\sigma,Z,X_0)$ up to its explosion time $e(X)$. Moreover, if $X,Y$ are two solutions up to $\tau$ and $\eta$ respectively, then $X_{t \wedge \tau \wedge \eta} = Y_{t \wedge \tau \wedge \eta}$. In particular, if $X$ is a solution up to $e(X)$, then $\eta \leq e(X)$ and $X_{t \wedge \tau} = Y_{t \wedge \tau}$.
\end{thm}

\begin{thm}
    Suppose $\sigma$ and $b$ are locally Lipschitz. Then the weak uniqueness holds for
    \begin{equation*}
    	X_t = X_0 + \int_0^t \sigma(X_s) dW_s + \int_0^t b(X_s)ds,
    \end{equation*}
    where $W$ is a Brownian motion.
\end{thm}

\paragraph{Stratonovich integral.} For two continuous semimartingales $H,Z$, the Stratonovich integral is defined as
\begin{equation*}
	\int_0^t H_s \circ dZ_s = \int_0^tH_sdZ_s + \frac{1}{2}\inn{H,Z}_t.
\end{equation*}
In such case, the It\^o formula becomes
\begin{equation*}
	df(X_t) = \frac{\partial f}{\partial x^\alpha}(Z_t) \circ dZ_t.
\end{equation*}

Suppose that $V_{\alpha}$, $\alpha = 1,2,\cdots, \ell$ are smooth vector fields in $\R^N$, that is,
\begin{equation*}
	V_{\alpha} = (V_{\alpha}^1,\cdots,V_{\alpha}^N) \colon \R^N \sto \R^N
\end{equation*}
and vector fields means that for any $f \in C^1(\R^N)$,
\begin{equation*}
	V_{\alpha}f = \sum_{i=1}^N V_{\alpha}^i\frac{\partial f}{\partial x^i}.
\end{equation*}
More generally, if
\begin{equation*}
	X_t=X_0+\int_0^t V_\alpha\left(X_s\right) \circ d Z^\alpha_s,
\end{equation*}
then
\begin{equation*}
	X_t=X_0+\int_0^t V_\alpha\left(X_s\right) d Z_s^\alpha+\frac{1}{2} \int_0^t \nabla_{V_\beta} V_\alpha\left(X_s\right) d\left\langle Z^\alpha, Z^\beta\right\rangle_s.
\end{equation*}
\begin{proof}
    By definition,
    \begin{equation*}
    	dX_t = V_\alpha(X_t) \circ dZ^\alpha_t = V_\alpha(X_t)dZ^\alpha_t + \frac{1}{2}d \inn{V_\alpha(X_t),Z^\alpha_t}.
    \end{equation*}
    or
    \begin{equation*}
    	dX^i_t = \sum_{\alpha=1}^\ell V^i_\alpha(X_t)dZ^\alpha_t + \frac{1}{2} \sum_{\alpha=1}^\ell d \inn{V^i_\alpha(X_t),Z^\alpha_t}.
    \end{equation*}
    For $d\inn{V^i_\alpha(X_t),Z^\alpha_t}$, first because $\sum_{\alpha} V^i_\alpha(X_t)dZ^\alpha_t$ is the local martingale part of $dX^i_t$,
    \begin{equation*}
    	d\inn{X^i,Z^\alpha}_t = \sum_{\beta=1}^\ell V^i_\beta(X_t) d \inn{Z^\beta,Z^\alpha}_t.
    \end{equation*}
    Then by It\^o formula,
    \begin{equation*}
    	dV^i_\alpha(X_t) = \sum_{j=1}^n \frac{\partial V^i_\alpha}{\partial x^j}dX_t^j + \text{ bounded variation part }.
    \end{equation*}
    So
    \begin{align*}
    	d\inn{V^i_\alpha(X_t),Z^\alpha_t} &= \sum_{j=1}^n \frac{\partial V^i_\alpha}{\partial x^j} d \inn{X^j, Z^\alpha}_t\\
    	&= \sum_{\beta = 1}^\ell \sum_{j=1}^n V_\beta^j\frac{\partial V^i_\alpha}{\partial x^j} d \inn{Z^\beta,Z^\alpha}_t\\
    	&= \nabla_{V_\beta} V^i_\alpha d \inn{Z^\beta,Z^\alpha}_t.
    \end{align*}
    Therefore,
    \begin{equation*}
    	dX_t = V_\alpha(X_t)dZ^\alpha_t + \frac{1}{2}\nabla_{V_\beta} V_\alpha (X_t) d \inn{Z^\beta,Z^\alpha}_t. \qedhere
    \end{equation*}
\end{proof}
In the sense of Stratonovich integral, if $X_t=X_0+\int_0^t V_\alpha\left(X_s\right) \circ d Z^\alpha_s$ and $f \in C^2(\R^N)$, then
\begin{equation*}
	f(X_t) = f(X_0) + \int_0^t (V_\alpha f)(X_s) \circ dZ^\alpha_s.
\end{equation*}
In differential form, if $dX_t = V_\alpha(X_t) \circ dZ^\alpha_t$
\begin{equation*}
	df(X_t) = (V_\alpha f)(X_t) dZ^\alpha_t.
\end{equation*}

\section{SDE on Manifolds}

\begin{defn}
    Let $M$ be a smooth manifold. A continuous $M$-valued process $X$ is called a $M$-valued semi-martingale if $f(X)$ is a real-valued semi-martingale for all $f \in C^\infty(M)$.
\end{defn}
\begin{rmk}
    We can also define it up to a stopping time $\tau$ as before. By It\^o formula, it is obviously that $M = \R^N$ gives the usual definition of semimartingales. Moreover, the test function space can be chosen as $C_c^\infty(M)$.
\end{rmk}

Let $V_1,\cdots,V_\ell \in \Gamma(TM)$, $Z$ be an $\R^\ell$-valued semimartingale, and an $M$-valued random variable $X_0 \in \mathcal{F}_0$. Consider an equation symbolically written as
\begin{equation*}
	dX_t = V_\alpha \circ dZ^\alpha_t,
\end{equation*}
and denoted it as $\op{SDE}(V_1,\cdots,V_\ell;Z,X_0)$.

\begin{defn}
    An $M$-valued semimartingale $X$ defined up to a stopping time $\tau$ is a solution of $\op{SDE}(V_1,\cdots,V_\ell;Z,X_0)$ up to $\tau$ if for all $f \in C^\infty(M)$,
    \begin{equation*}
    	f(X_t) = f(X_0) + \int_0^t(V_\alpha f)(X_s) \circ dZ_s^\alpha.
    \end{equation*}
\end{defn}

\begin{prop}
    Suppose $\Phi \colon M \sto N$ is a diffeomorphism and $X$ a solution of $\op{SDE}(V_1,\cdots,V_\ell;Z,X_0)$. Then $\Phi(X)$ is a solution of $\op{SDE}(\Phi_* V_1,\cdots,\Phi_* V_\ell;Z,\Phi(X_0))$ on $N$.
\end{prop}
\begin{proof}
    Let $Y = \Phi(X)$. For any $f \in C^\infty(N)$, because $f \circ \Phi \in C^\infty(M)$ and $X$ is a solution of $\op{SDE}(V_1,\cdots,V_\ell;Z,X_0)$,
    \begin{equation*}
    	f \circ \Phi(X_t) = f \circ \Phi(X_0) + \int_0^t V_\alpha (f \circ \Phi)(X_s) \circ dZ_s^\alpha.
    \end{equation*}
    Note that $\Phi_* \colon \Gamma(TM) \sto \Gamma(TN)$ such that for and $f \in C^\infty(N)$,
    \begin{equation*}
    	\Phi_*V(f)(\Phi(x)) = V(f\circ \Phi)(x).
    \end{equation*}
    It follows that
    \begin{equation*}
    	f(Y_t) = f(Y_0) + \int_0^t \Phi_*V_\alpha (f)(Y_s) \circ dZ_s^\alpha,
    \end{equation*}
    which means that $Y$ is a solution of $\op{SDE}(\Phi_* V_1,\cdots,\Phi_* V_\ell;Z,\Phi(X_0))$.
\end{proof}

Note that by Whitney's embedding theorem, any smooth manifold $M$ can be properly embedded in a Euclidean space, i.e., $i \colon M \hookrightarrow \R^N$, where $i$ is a smooth proper map. Therefore, $i(M)$ is a closed submanifold in $\R^N$. Note that here closed submanifold means that a submanifold that is a closed subset. Therefore, we can always view $M \subset \R^N$ as an embedded submanifold that is closed. Moreover, we usually assume $M$ without boundary. When $M$ is non-compact, we can consider its one-point compactification $\widehat{M} = M \cup \bb{\partial_M}$. So $\bb{x_n}_n \subset M$ such that $x_n \sto \partial_M$ in $\widehat{M}$ if and only if $\norm{x_n}_{\R^N} \sto \infty$.

In the following we always assumed that $M \subset \R^N$ is a submanifold that is closed and without boundary. Moreover, let
\begin{equation*}
    i = (f^1,\cdots,f^N) \colon M \hookrightarrow \R^N
\end{equation*}
and $f^i$ are coordinate functions that $f^i(x) = x^i$.

\begin{prop}
    Suppose $M \subset \R^N$ is a submanifold that is closed. Let $f^1,\cdots,f^N$ be coordinate functions. Let $X$ be an $M$-valued continuous process.
    \begin{enumerate}[label=(\arabic{*})]
        \item $X$ is a semimartingale on $M$ if and only if it is an $\R^N$-valued semimartingale, i.e. $f^i(X)$ is a $\R$-valued semimartingale for each $i = 1,2,\cdots,N$.

        \item $X$ is a solution of $\op{SDE}(V_1,\cdots,V_\ell;Z,X_0)$ up to a stopping time $\sigma$ if and only if for each $i = 1,2,\cdots,N$,
        \begin{equation*}
            f^i(X_t) = f^i(X_0) + \int_0^t (V_\alpha f^i)(X_s) \circ dZ_s^\alpha,\quad 0\leq t < \sigma.
        \end{equation*}
    \end{enumerate}
\end{prop}
\begin{proof}
    \begin{enumerate}[label=(\arabic{*})]
        \item $\Rightarrow:$ It is by the definition.

        $\Leftarrow:$ For any $f \in C^\infty(M)$, since $M$ is closed $\R^N$, it can be extended on $\tilde{f} \in C^\infty(\R^N)$ such that $f(X) = \tilde{f}(X)$. Because $X$ is $\R$-valued semimartingale, $f(X) = \tilde{f}(X)$ is also a semimartingale.

        \item $\Rightarrow:$ It is by the definition.

        $\Leftarrow:$ For any $f \in C^\infty(M)$, let $\tilde{f} \in C^\infty(\R^N)$ be an extension. Then
        \begin{equation*}
            f(X_t) = \tilde{f}\bc{f^1(X_t),\cdots,f^N(X_t)}.
        \end{equation*}
        Therefore,
        \begin{align*}
            df(X_t) &= \tilde{f}\bc{f^1(X_t),\cdots,f^N(X_t)} \circ d(f^i(X_t)) \\
            &= \tilde{f}\bc{f^1(X_t),\cdots,f^N(X_t)} \circ V_\alpha f^i(X_t) \circ dZ_t^\alpha \\
            &= \bc{\tilde{f}\bc{f^1(X_t),\cdots,f^N(X_t)}V_\alpha f^i(X_t)} \circ dZ_t^\alpha \\
            &= V_{\alpha} \tilde{f}(X_t) \circ dZ^\alpha_t = V_{\alpha} f(X_t) \circ dZ^\alpha_t,
        \end{align*}
        where the final equality is because $V_\alpha \in \Gamma{TM}$ and $\tilde{f}$ is an extension of $f$. \qedhere
    \end{enumerate}
\end{proof}
\begin{rmk}
    Note that for $f \in \R^k \sto \R$ and $g \colon \R^n \sto \R^k$
    \begin{align*}
        d (f \circ g(M)) &= f_{x_i}(g(M)) \circ dg^i(M) \\
        &= f_{x_i}(g(M)) \circ g^i_{x_j}(M) \circ dM^j \\
        &= \nabla f(g(M))^\top Dg(M) \circ dM
    \end{align*}
    under Stratonovich integral. In fact, if $H,X,W$ are continuous semimartingales, we have
    \begin{equation*}
        H \circ d \bc{\int X \circ dW} = (HX) \circ dW,
    \end{equation*}
    because
    \begin{equation*}
        d\inn{HX,W} = Hd\inn{X,W} + Xd\inn{H,W},
    \end{equation*}
    which is because of It\^o formula, $d(HX) = HdX +XdH + d\inn{H,X}$.
\end{rmk}
\begin{rmk}
    If $X$ is a solution of $\op{SDE}(V_1,\cdots,V_\ell;Z,X_0)$ up to its explosion time $e(X)$ and $X_0 \in M$, then $X_t \in M$ for $0 \leq t < e(X)$. Moreover, $\op{SDE}(V_1,\cdots,V_\ell;Z,X_0)$ is unique up its explosion time.
\end{rmk}

\section{Diffusion Process}

Let $L$ be a smooth second order elliptic, but not necessarily non degenerate, differential operator on a smooth manifold $M$, that is, 
\begin{equation*}
    L \colon C^\infty(M) \rightarrow C^\infty(M)
\end{equation*}
such that on every chart $(U;x^1,\cdots,x^d)$
\begin{equation*}
    Lf(x) = a^{ij}(x) \frac{\partial^2}{\partial x^i \partial x^j} f(x) + b^i(x)\frac{\partial}{\partial x^i}f(x) + c(x)f(x)
\end{equation*}
for some $a^{ij},b^i,c \in C^\infty(U)$ and $\sum_{i,j}a^{ij}(x)\xi_i\xi_j \geq 0$.
\begin{rmk}
    If $M$ is equipped with a torsion-free connection $\nabla$, like a Riemannian manifold, then $L$ can be written as a coordinate-free formulation, that is,
    \begin{equation*}
        Lf \defeq \inn{A,\nabla^2 f} + \inn{b,\nabla f}+cf,
    \end{equation*}
    for $b \in \Gamma(TM)$, $c \in C^\infty(M)$, and $A$, a symmetric $(2,0)$-tensor, where $\inn{\cdot,\cdot}$ is the natural dual operation.
\end{rmk}
For $f \in C^2(M)$ and $\omega \in W(M)$, let
\begin{equation*}
    M^f(\omega)_t = f(\omega_t) - f(\omega_0) - \int_0^t Lf(\omega_s)ds.
\end{equation*}

\begin{defn}
    An $\mathbb{F}$-adapted process $X$ that is valued in $M$ is called a diffusion process generated by $L$ if $X$ is a $M$-valued semimartingale up to $e(X)$ and 
    \begin{equation*}
        M^f(X)_t = f(X_t) - f(X_0) - \int_0^tLf(X_s)ds,\quad 0 \leq t < e(X),
    \end{equation*}
    is a local martingale for all $f \in C^\infty(M)$.

    \item A probability measure $\mu$ defined on $(W(M),\mathcal{F}_{M,\infty})$ is called a diffusion measure generated by $L$ if
    \begin{equation*}
        M^f(\omega)_t=f\left(\omega_t\right)-f\left(\omega_0\right)-\int_0^t L f\left(\omega_s\right) d s, \quad 0 \leq t<e(\omega),
    \end{equation*}
    is a local $\mathbb{F}_M$ local martingale for all $f \in C^\infty(M)$.
\end{defn}
\begin{rmk}
    If $X$ is an $L$-diffusion, then $\mu^X =X_{\#}\Pb$ is an $L$-diffusion measure. Conversely, if $\mu$ is an $L$-diffusion measure, then the coordinate process is an $L$-diffusion process.
\end{rmk}

For given $L$, assume locally
\begin{equation*}
    L=\frac{1}{2} \sum_{i, j=1}^d a^{i j}(x) \frac{\partial^2}{\partial x^i \partial x^j}+\sum_{i=1}^d b^i(x) \frac{\partial}{\partial x^i},
\end{equation*}
where $(a^{ij}(x))$ is positive semi-definite. Define
\begin{equation*}
    \Gamma(f,g) = L(fg) - fLg - gLf,\quad f,g \in C^\infty(M).
\end{equation*}
So locally
\begin{equation*}
    \Gamma(f,g) = a^{ij}\frac{\partial f}{\partial x^i} \frac{\partial f}{\partial x^j}.
\end{equation*}
And the semi-elliptic means that for any $f \in C^\infty(M)$, $\Gamma(f,f) \geq 0$. Combining this with bilinearity, it implies that if $g^1,\cdots,g^n$, the matrix $(\Gamma(g^i,g^j))$ is positive semimartingale.

Assume $M$ is a submanifold of $\R^N$, which is closed. Let $z = (z^1,\cdots,z^N)$ be coordinates in $\R^N$ and $f^\alpha$ ($\alpha = 1,2,\cdots, N$) are coordinate maps $M \sto \R^N$ by $f^\alpha(z) = z^\alpha$. Let
\begin{equation*}
    \tilde{a}^{\alpha \beta} = \Gamma(f^\alpha,f^\beta),\quad \tilde{b}^\alpha = Lf^\alpha.
\end{equation*}
Then they are in $C^\infty(M)$ and $(\tilde{a}^{\alpha \beta})$ is positive semi-definite. By the closedness of $M$, $\tilde{a},\tilde{b}$ can be extended to $\R^N$ and let
\begin{equation*}
    \tilde{L}=\frac{1}{2} \sum_{\alpha, \beta=1}^N \tilde{a}^{\alpha \beta} \frac{\partial^2}{\partial z^\alpha \partial z^\beta}+\sum_{\alpha=1}^N \tilde{b}^\alpha \frac{\partial}{\partial z^\alpha}
\end{equation*}
be defined on $\R^N$. It is a true extension of $L$.

\begin{lem}
    Suppose $f \in C^\infty(M)$. Then for any $\tilde{f} \in C^\infty(\R^N)$ which extends $f$ from $M$ to $\R^N$, $\tilde{L}\tilde{f} = Lf$ on $M$.
\end{lem}
\begin{proof}
    Let $(x^1,\cdots,x^d)$ be a local coordinate on $M$. Note that
    \begin{equation*}
        f(x) = \tilde{f}(f^1(x),\cdots,f^N(x)).
    \end{equation*}
    Then
    \begin{align*}
        Lf(x) &= L \tilde{f}(f^1(x),\cdots,f^N(x)) \\
        &= \frac{1}{2} a^{i j}(x) \frac{\partial^2 \tilde{f}(f^1(x),\cdots,f^N(x))}{\partial x^i \partial x^j}+ b^i(x) \frac{\partial \tilde{f}(f^1(x),\cdots,f^N(x))}{\partial x^i} \\
        &= \tilde{L}\tilde{f}(z).
    \end{align*}
\end{proof}

For $\tilde{L}$ on $\R^n$, if $\tilde{a} = \tilde{\sigma}\tilde{\sigma}^\top$, it is the generator of diffusion process that satisfies
\begin{equation*}
    X_t = X_0 + \int_0^t \tilde{\sigma}(X_s)dW_s + \int_0^t \tilde{b}(X_s)ds.
\end{equation*}
We can clearly choose $\tilde{\sigma} = \tilde{a}^{\frac{1}{2}}$ and $W$ a standard Brownian motion in $\R^N$ independent of $X_0$.
\begin{rmk}
    The existence of solution up to a stopping time needs the local Lipschitz of $\tilde{\sigma}$, which can be guaranteed by the local Lipschitz of $\tilde{a}$.
\end{rmk}

Moreover, for such diffusion process $X$ on $\R^N$, $\mu^X = X_{\#}\Pb$, the distribution on $W(\R^N)$, is concentrated on $W(M)$, i.e., $X$ is in fact a $L$-diffusion on $M$. In fact, we have the following theorem.

\begin{thm}
    If $L$ is a smooth second order semi-elliptic operator a $C^\infty$-manifold $M$ and $\mu_0$ is a probability measure on $M$, then there exists a unique $L$-diffusion measure with initial distribution $\mu_0$.
\end{thm}


